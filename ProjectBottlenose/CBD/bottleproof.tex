\documentclass[12pt]{article}
\usepackage{color,amsmath,amssymb,amsfonts,amsthm,fullpage,setspace}
\renewcommand{\qedsymbol}{$\blacksquare$}
\DeclareMathOperator*{\argmin}{\text{argmin}}
\DeclareMathOperator{\e}{\text{E}}
\DeclareMathOperator{\var}{\text{Var}}
\newtheorem{lemma}{Lemma}
\newtheorem{proposition}{Proposition}
\begin{document}
\onehalfspacing
There are two players, sender and receiver, and a state of nature $q\sim U[0,1]$ which is known to the sender but not to the receiver. Let $g$ denote the density of $q$. The sender and receiver have preferences over actions $U(q,A)=-(A-q)^{2}$. Let $\mathcal{M}$ denote the set of all non-decreasing functions from $[0,1]$ to $[0,1]$. Let $m\in\mathcal{M}$ be a message function. Let $Q(\widetilde{m})$ denote the set of all $q\in[0,1]$ such that for some $e\in[-\bar{\epsilon},\bar{\epsilon}]$, $\widetilde{m}=m(q)+e$ . Finally, put $q_{+}(\widetilde{m})\equiv\sup Q(\widetilde{m})$, and $q_{-}(\widetilde{m})\equiv\inf Q(\widetilde{m})$. Suppose that the receiver receives the message $\widetilde{m}\in[-\bar{\epsilon},1+\bar{\epsilon}]$. $q|\widetilde{m}$ has support $[q_{-}(\widetilde{m}),q_{+}(\widetilde{m})]$, and 
\begin{equation}
g(q|\widetilde{m})=\frac{f_{e}(\widetilde{m}-m(q))\mathbf{1}_{0\leq q\leq 1}}{\int_{0}^{1}{f_{e}(\widetilde{m}-m(t))\mathbf{1}_{0\leq t\leq 1}dt}}.
\end{equation}
Her optimal action is 
\begin{equation}
A(\widetilde{m})\equiv\argmin_{a}\int_{q_{-}(\widetilde{m})}^{q_{+}(\widetilde{m})}{(q-a)^{2}g(q|\widetilde{m})}=\int_{q_{-}(\widetilde{m})}^{q_{+}(\widetilde{m})}{qg(q|\widetilde{m})}
\end{equation}
\noindent As it will appear often, let the cost functional $C:\mathcal{M}\rightarrow\mathbb{R}$ be given by
\begin{equation}\label{eq:costfunctional}
C[m]\equiv\int_{0}^{1}{\int_{-\bar{\epsilon}}^{\bar{\epsilon}}{(q-A(m(q)+e))^{2}f_{\epsilon}(e)de}dq}.
\end{equation}
The \textit{sender's problem} is to choose a message $m\in\mathcal{M}$ that minimizes $C$. A change of variables and an application of Fubini's Theorem yeild
\begin{align}
C[m]&=\int_{0}^{1}{\int_{m(q)-\bar{\epsilon}}^{m(q)+\bar{\epsilon}}{(q-A(\widetilde{m}))^{2}f_{\epsilon}(\widetilde{m}-m(q))d\widetilde{m}}dq}\\
&=\int_{-\bar{\epsilon}}^{1+\bar{\epsilon}}{\int_{q_{-}(\widetilde{m})}^{q_{+}(\widetilde{m})}{(q-A(\widetilde{m}))^{2}f_{\epsilon}(\widetilde{m}-m(q))dq}d\widetilde{m}}.
\end{align}
Finally, note that $\int_{-\bar{\epsilon}}^{1+\bar{\epsilon}}{\int_{q_{-}(\widetilde{m})}^{q_{+}(\widetilde{m})}{dq}d\widetilde{m}}=\int_{0}^{1}{\int_{m(q)-\bar{\epsilon}}^{m(q)+\bar{\epsilon}}{d\widetilde{m}}dq}=2\bar{\epsilon}$.

%%%%%%%%%%%%%%%%%%%%%%%%%%%%%%%%%%%%%%%
%%% PROPOSITION: DISCRETEISOPTIMAL %%%%
%%%%%%%%%%%%%%%%%%%%%%%%%%%%%%%%%%%%%%%
\pagebreak
\begin{proposition}\label{prop:discreteisoptimal}
If the error is uniform, then the discrete message function is optimal. 
\end{proposition}
\begin{proof}
$f_{e}(e)=\tfrac{1}{2\bar{\epsilon}}\mathbf{1}_{-\bar{\epsilon}\leq e\leq\bar{\epsilon}}$, $g(q|\widetilde{m})=\tfrac{1}{q_{+}(\widetilde{m})-q_{-}(\widetilde{m})}\mathbf{1}_{q_{-}(\widetilde{m})\leq q\leq q_{+}(\widetilde{m})}$, $A(\widetilde{m})=\tfrac{q_{+}(\widetilde{m})+q_{-}(\widetilde{m})}{2}$, and
\begin{equation}
C[m]=\frac{1}{2\bar{\epsilon}}\int_{-\bar{\epsilon}}^{1+\bar{\epsilon}}{\left[\frac{2}{3}\left(\frac{q_{+}(\widetilde{m})-q_{-}(\widetilde{m})}{2}\right)^{3}\right]d\widetilde{m}}.
\end{equation}
Under $m_{d}$, $q_{+}(\widetilde{m})-q_{-}(\widetilde{m})=\tfrac{2\bar{\epsilon}}{2\bar{\epsilon}+1}$, and hence $C[m_{d}]=\tfrac{1}{3}\left(\tfrac{\bar{\epsilon}}{1+2\bar{\epsilon}}\right)^{2}$. Let $m\in\mathcal{M}$.
\begin{align}
C[m]&=\frac{1+2\bar{\epsilon}}{2\bar{\epsilon}}\int_{-\bar{\epsilon}}^{1+\bar{\epsilon}}{\left[\frac{2}{3}\left(\frac{q_{+}(\widetilde{m})-q_{-}(\widetilde{m})}{2}\right)^{3}\right]\cdot\frac{d\widetilde{m}}{1+2\bar{\epsilon}}}\\
&\geq\frac{1+2\bar{\epsilon}}{2\bar{\epsilon}}\cdot\frac{2}{3}\left(\int_{-\bar{\epsilon}}^{1+\bar{\epsilon}}{\left(\frac{q_{+}(\widetilde{m})-q_{-}(\widetilde{m})}{2}\right)\cdot\frac{d\widetilde{m}}{1+2\bar{\epsilon}}}\right)^{3}\\
&=\frac{1+2\bar{\epsilon}}{3\bar{\epsilon}}\left(\frac{\bar{\epsilon}}{1+2\bar{\epsilon}}\right)^{3}\\
&=C[m_{d}],
\end{align}
where the second line follows by Jensen's inequality.
\end{proof}

%%%%%%%%%%%%%%%%%%%%%%%%%%%%%%%%%%%%%%%%%%
%%% PROPOSITION: DISCRETEISNOTOPTIMAL %%%%
%%%%%%%%%%%%%%%%%%%%%%%%%%%%%%%%%%%%%%%%%%
\begin{proposition}\label{prop:discreteisnotoptimal}
If the error is not uniform and if its variance is sufficiently small $(<C[m_{d}]/(1+2\bar{\epsilon}))$, then the optimal message function is not discrete. 
\end{proposition}
\begin{proof}
Put $m_{i}(q)\equiv q$. Under $m_{i}$, we have that $q_{-}(\widetilde{m})=\widetilde{m}-\bar{\epsilon}$, $q_{+}(\widetilde{m})=\widetilde{m}+\bar{\epsilon}$, and $g(q|\widetilde{m})=f_{e}(\widetilde{m}-m(q))$. Put $\delta\equiv\widetilde{m}-m_{i}(q)$. $A(\widetilde{m})=\widetilde{m}-\int_{-\bar{\epsilon}}^{\bar{\epsilon}}{\delta f_{e}(\delta)d\delta}$, and 
\begin{equation}
C[m_{i}]=\int_{-\bar{\epsilon}}^{1+\bar{\epsilon}}{\int_{-\bar{\epsilon}}^{\bar{\epsilon}}{\left(\delta-\int_{-\bar{\epsilon}}^{\bar{\epsilon}}{\delta' f_{e}(\delta')d\delta'}\right)^{2}f_{\epsilon}(\delta)d\delta}d\widetilde{m}}\leq C[m_{d}]
\end{equation}
from which the result obtains.
\end{proof}

\end{document}