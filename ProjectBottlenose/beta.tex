\documentclass[12pt]{article}
\usepackage{amsmath,amssymb,amsfonts,fullpage,setspace}
\begin{document}
\begin{center}
\textbf{Appendix: Numerical Solution with Non-Uniformly Distributed Errors}
\end{center}
\onehalfspacing
We look for solutions in the space of increasing piecewise-continuous functions on $[0,1]$, which we called $M$. The sender's problem is intractable. We therefore approximate the solution by a step-function. We have two choices. First, we can establish a necessary condition for optimality, namely, the Euler-Lagrange equation, and hope that a solution exists. We can then numerically approximate the solution to the Euler-Lagrange equation with $\ell_{0}(x)=x$: 
\begin{equation}
a_{n}(m)=E[q|\ell_{n}(\max\{m-\bar{\epsilon},0\})\leq q\leq\ell_{n}(\min\{m+\bar{\epsilon},1\})]
\end{equation}
where the expectation is taken with respect to the PDF $g$. The integral can be written as 
\begin{equation}
D_{n}(m)=(\ell_{n}(m)-a(m+\bar{\epsilon}))^{2}f(\bar{\epsilon})-(\ell_{n}(m)-a(m-\bar{\epsilon}))^{2}f(-\bar{\epsilon})
\end{equation}
Define
\begin{equation}
A_{n}^{k}(m)=\int_{m-\bar{\epsilon}}^{m+\bar{\epsilon}}{a_{n}(\tilde{m})^{k}f'(\tilde{m}-m)d\tilde{m}}
\end{equation}
We iterate according to
\begin{equation}
A_{n}^{k=0}(m)\ell_{n+1}^{2}(m)-2A_{n}^{k=1}(m)\ell_{n+1}(m)+A_{n}^{k=2}(m)-D_{n}(m,\ell_{n}(m))=0
\end{equation}
Second, we can look for solutions in the space of step-functions on $[0,1]$. Such a step-function should \textit{not} be confused with the step-function obtained as a solution. Step-functions are dense in $M$: for each $m\in M$ and $\epsilon>0$, there is an $n\geq0$ such that $||m-m_{n}||_{\infty}<0$. Step-functions make for robust approximations, as they allow for discontinuities in the solution (a feature which we find in equilibrium). For $n\geq 1$, $m_{n}:[0,1]\rightarrow\{0,\alpha_{1},\alpha_{2},\ldots,a_{n-1},1\}$ be given by
\begin{equation}
m_{n}(q)=\sum_{j=0}^{n-1}{\frac{j}{n}\cdot\mathbf{1}_{\alpha_{j}\leq q<\alpha{j+1}}}
\end{equation} 
Now $m_{n}:\{0,\alpha_{1},\alpha_{2},\ldots,a_{n-1},1\}\rightarrow[0,1]$ will not be sufficient for our uses, since the receiver receives an element of the codomain of $m_{n}$ \textit{plus} noise. Let $\tilde{m}\mapsto\ell(\tilde{m})\subset[0,1]$ be given by
\begin{equation}
\ell_{n}(p)=\left[\sum_{j=0}^{n-1}{\frac{j}{n}\cdot\mathbf{1}_{\alpha_{j}\leq p<\alpha_{j+1}}},\sum_{j=0}^{n-1}{\frac{j+1}{n}\cdot\mathbf{1}_{\alpha_{j}\leq p<\alpha_{j+1}}}\right)
\end{equation}
Now
\begin{equation}
q_{+}(\tilde{m})=\sup\ell_{n}(\min\{\tilde{m}+\bar{\epsilon},1\})=\sum_{j=0}^{n-1}{\frac{j+1}{n}\cdot\mathbf{1}_{\alpha_{j}\leq\min\{\tilde{m}+\bar{\epsilon},1\}<\alpha_{j+1}}}.
\end{equation}
Similarly,
\begin{equation}
q_{-}(\tilde{m})=\inf\ell_{n}(\max\{\tilde{m}+\bar{\epsilon},1\})=\sum_{j=0}^{n-1}{\frac{j}{n}\cdot\mathbf{1}_{\alpha_{j}\leq\max\{\tilde{m}-\bar{\epsilon},0\}<\alpha_{j+1}}}.
\end{equation}
The sender's problem is to choose a vector $a=(0,\alpha_{1},\alpha_{2},\ldots,\alpha_{n-1},1)$ to
\begin{equation}
\min_{a}\int_{0}^{1}{\int_{-\bar{\epsilon}}^{\bar{\epsilon}}{\left(q-\frac{\overline{q}(m(q)+e+\bar{\epsilon})+\underline{q}(m(q)+e-\bar{\epsilon})}{2}\right)^{2}f_{e}(e)de}dq}
\end{equation}
Fix $j\in\{0,1,\ldots,n\}$. Note that $m$ is constant on $[q_{j},q_{j+1}]$. For $e\in[a,b]$, define
\begin{align}
T_{A}(a,b)&=\int_{q_{j}}^{q_{j+1}}{q^{2}g(q)dq}\cdot\int_{a}^{b}{f(e)de}\\
T_{B}(a,b)&=\int_{q_{j}}^{q_{j+1}}{qg(q)dq}\cdot\int_{a}^{b}{A(m+e)f(e)de}\\
T_{C}(a,b)&=\int_{q_{j}}^{q_{j+1}}{g(q)dq}\cdot\int_{a}^{b}{A(m+e)^{2}f(e)de}
\end{align} 
so that
\begin{equation}
\int_{q_{j}}^{q_{j+1}}{\int_{a}^{b}{(q-A(m(q)-e))^{2}f(e)de}g(q)dq}=T_{A}(a,b)-2T_{B}(a,b)+T_{C}(a,b)
\end{equation}
\end{document}