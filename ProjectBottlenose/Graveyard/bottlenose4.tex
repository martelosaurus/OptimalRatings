\documentclass[12pt]{article}
\usepackage[margin=.5in]{geometry}
\usepackage{amsmath}
\usepackage{amsthm}
\usepackage{amsfonts}
\usepackage{amssymb}
\usepackage{mathtools}
\usepackage{float}
\usepackage{graphicx}
\usepackage{setspace}
\usepackage{epstopdf}
\usepackage[utf8]{inputenc}
\usepackage[T1]{fontenc}
\usepackage{lmodern}

\newtheorem{assumption}{Assumption}
\newtheorem{lemma}{Lemma}

\DeclareMathOperator*{\argmin}{argmin}
\DeclareMathOperator*{\argmax}{argmax}

\begin{document}
% discrete messages
\section{discrete actions}
Let $m:[0,1]\rightarrow[0,1]$ be the sender's message function. Let the receiver's \textbf{discrete} action $A:[0,1]\rightarrow\{0,1\}$ be given by
\begin{equation*}
A(\tilde{m})=\left\{\begin{array}{ccc}
1 &\text{if}& m^{-1}(\tilde{m})>V\\
0 &\text{otherwise}&
\end{array}\right.
\end{equation*}
where $V$ is distributed according to $G$. The expected utility---with respect to $V$---for $e>0$ is
\begin{align*}
U_{+}(q,m(q),e)&=\underbrace{G(q)}_{A=0\text{ and }V<q}+\underbrace{(1-G(m^{-1}(m(q)+e)))}_{A=1\text{ and }V\geq q}\\ \\
&\approx G(q)+(1-G(q+m^{-1'}(m(q))e))\\
&=G(q)+(1-G(q+e/m'(q)))\\
&\approx G(q)+(1-G(q)-g(q)(e/m'(q)))\\
&=1-g(q)(e/m'(q))
\end{align*}
while the expected utility---again, with respect to $V$---for $e\leq0$ is 
\begin{align*}
U_{-}(q,m(q),e)&=\underbrace{G(m^{-1}(m(q)+e))}_{A=0\text{ and }V<q}+\underbrace{(1-G(q))}_{A=1\text{ and }V\geq q}\\ \\
&\approx G(q+m^{-1'}(m(q))e)+(1-G(q))\\
&=G(q+e/m'(q))+(1-G(q))\\
&\approx G(q)+g(q)(e/m'(q))+(1-G(q))\\
&=1+g(q)(e/m'(q)).
\end{align*}
The sender chooses $m$ to maximize total expected utility:
\begin{align*}
\min_{m}\:=&\int\limits_{0}^{1}{\left\{\:\int\limits_{-\bar{\epsilon}}^{0}{U_{-}(q,m(q),e)\left(\frac{de}{2\bar{\epsilon}}\right)}+\int\limits_{0}^{\bar{\epsilon}}{U_{+}(q,m(q),e)\left(\frac{de}{2\bar{\epsilon}}\right)}\right\}I(q)dq}\\
\approx&\int\limits_{0}^{1}{\left\{\:\int\limits_{-\bar{\epsilon}}^{0}{(1+g(q)(e/m'(q)))\left(\frac{de}{2\bar{\epsilon}}\right)}+\int\limits_{0}^{\bar{\epsilon}}{(1-g(q)(e/m'(q)))\left(\frac{de}{2\bar{\epsilon}}\right)}\right\}I(q)dq}\\
=&\int\limits_{0}^{1}{\left\{1+\int\limits_{-\bar{\epsilon}}^{0}{g(q)(e/m'(q))\left(\frac{de}{2\bar{\epsilon}}\right)}-\int\limits_{0}^{\bar{\epsilon}}{g(q)(e/m'(q))\left(\frac{de}{2\bar{\epsilon}}\right)}\right\}I(q)dq}\\
=&\int\limits_{0}^{1}{\left\{1-\frac{\bar{\epsilon}g(q)}{2m'(q)}\right\}I(q)dq}.
\end{align*}
The Euler-Lagrange equation reads
\begin{equation*}
\left\{1+\frac{\bar{\epsilon}g(q)}{2m'(q)^2}\right\}I(q)=K
\end{equation*}
for some constant $K$. Equivalently,
\begin{equation*}
\left\{2m'(q)^2+\bar{\epsilon}g(q)\right\}I(q)=2Km'(q)^2
\end{equation*}
or
\begin{equation*}
\bar{\epsilon}g(q)I(q)=2(K-I(q))m'(q)^2
\end{equation*}
or
\begin{equation*}
\frac{\bar{\epsilon}g(q)I(q)}{2(K-I(q))}=m'(q)^2
\end{equation*}
which implies that $K>I(q)$ for all $q\in[0,1]$. 
\end{document}