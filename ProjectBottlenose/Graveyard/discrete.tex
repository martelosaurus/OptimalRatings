%2multibyte Version: 5.50.0.2960 CodePage: 1252

\documentclass[12pt]{article}
%%%%%%%%%%%%%%%%%%%%%%%%%%%%%%%%%%%%%%%%%%%%%%%%%%%%%%%%%%%%%%%%%%%%%%%%%%%%%%%%%%%%%%%%%%%%%%%%%%%%%%%%%%%%%%%%%%%%%%%%%%%%%%%%%%%%%%%%%%%%%%%%%%%%%%%%%%%%%%%%%%%%%%%%%%%%%%%%%%%%%%%%%%%%%%%%%%%%%%%%%%%%%%%%%%%%%%%%%%%%%%%%%%%%%%%%%%%%%%%%%%%%%%%%%%%%
\usepackage{amsfonts}
\usepackage{mitpress}

%TCIDATA{OutputFilter=LATEX.DLL}
%TCIDATA{Version=5.50.0.2960}
%TCIDATA{Codepage=1252}
%TCIDATA{<META NAME="SaveForMode" CONTENT="1">}
%TCIDATA{BibliographyScheme=Manual}
%TCIDATA{Created=Wednesday, July 16, 2014 15:01:24}
%TCIDATA{LastRevised=Friday, January 29, 2016 13:42:54}
%TCIDATA{<META NAME="GraphicsSave" CONTENT="32">}
%TCIDATA{<META NAME="DocumentShell" CONTENT="Articles\SW\A Simple MIT Press Article">}
%TCIDATA{CSTFile=40 LaTeX article.cst}

\newtheorem{theorem}{Theorem}
\newtheorem{acknowledgement}[theorem]{Acknowledgement}
\newtheorem{algorithm}[theorem]{Algorithm}
\newtheorem{axiom}[theorem]{Axiom}
\newtheorem{case}[theorem]{Case}
\newtheorem{claim}[theorem]{Claim}
\newtheorem{conclusion}[theorem]{Conclusion}
\newtheorem{condition}[theorem]{Condition}
\newtheorem{conjecture}[theorem]{Conjecture}
\newtheorem{corollary}[theorem]{Corollary}
\newtheorem{criterion}[theorem]{Criterion}
\newtheorem{definition}[theorem]{Definition}
\newtheorem{example}[theorem]{Example}
\newtheorem{exercise}[theorem]{Exercise}
\newtheorem{lemma}[theorem]{Lemma}
\newtheorem{notation}[theorem]{Notation}
\newtheorem{problem}[theorem]{Problem}
\newtheorem{proposition}[theorem]{Proposition}
\newtheorem{remark}[theorem]{Remark}
\newtheorem{solution}[theorem]{Solution}
\newtheorem{summary}[theorem]{Summary}
\newenvironment{proof}[1][Proof]{\noindent\textbf{#1.} }{\ \rule{0.5em}{0.5em}}
\newdimen\dummy
\dummy=\oddsidemargin
\addtolength{\dummy}{72pt}
\marginparwidth=.5\dummy
\marginparsep=.1\dummy
\input{tcilatex}
\begin{document}

\title{A Simple MIT Press Style Article}
\author{A. U. Thor \\
%EndAName
At this Address}
\maketitle

\begin{abstract}
Replace this text with the text of your own abstract.
\end{abstract}

Now consider an alternative utility function. The receiver has a threshold
for quality $V$ so that she receives a payoff of 1 if $q>V$ and $A=1$ or if $%
q<V$ and $A=0$. She receives a payoff of 0 if $q<V$ and $A=1$ or if $q>V$
and $A=0$. The threshold for any given receiver is unknown, and distributed
according to $G$, with density $g$. An alternative interpretation is that $V$
is known, but there are many receivers receiving the same signal, and their
values of $V$ are distributed according to $G$. An additional alternative
interpretation is that $q$ is itself a signal of true quality. $V$ is known,
and there are multiple reviewers, whose reviews are correlated with the
signal $q$. This means that the threshold value of $q$ such that the actor
chooses $A=1$ or $A=0$ is a random variable with distribution $G$. 

Suppose that the optimal response function is $A(\widetilde{m})=1$ if $%
m^{-1}(\widetilde{m})>V$ and $A(\widetilde{m})=0$ if $m^{-1}(\widetilde{m}%
)<V $. We can calculate the expected utility given $q$ as comprised of three
components. First, if $\widetilde{m}<m(V)-\overline{\varepsilon }$, then the
receiver will always take the action $A=0$. If $\widetilde{m}>m(V)+\overline{%
\varepsilon }$, then she will always take action $A=1$. Only if $m\in
\lbrack m(V)-\overline{\varepsilon },m(V)+\overline{\varepsilon }]$ is there
a question of which action she will take.

Because $\varepsilon $ is uniform over $[-\overline{\varepsilon },\overline{%
\varepsilon }]$, and because $m(q)$ is locally linear, the probability that
she takes the correct action is $pr(A=0\mid V>q)=pr(A=1\mid V<q)=\frac{%
\left( V-q\right) m^{\prime }(q)+\overline{\varepsilon }}{2\overline{%
\varepsilon }}$ for small $\varepsilon $ and $V\in $ $[q-\frac{\overline{%
\varepsilon }}{m^{\prime }(q)},q+\frac{\overline{\varepsilon }}{m^{\prime
}(q)}]$. For a given message function $m^{\prime }(q)$, and a given $q$, the
expected utility is: \bigskip

\begin{eqnarray*}
E(U &\mid &q,m^{\prime }(q))=I(q)\int\nolimits_{0}^{q-\frac{\overline{%
\varepsilon }}{m^{\prime }(q)}}g(V)dV \\
&&+I(q)\int\nolimits_{q-\frac{\overline{\varepsilon }}{m^{\prime }(q)}}^{q}%
\frac{\left( q-V\right) m^{\prime }(q)+\overline{\varepsilon }}{2\overline{%
\varepsilon }}g(V)dV \\
&&+I(q)\int\nolimits_{q}^{q+\frac{\overline{\varepsilon }}{m^{\prime }(q)}}%
\frac{\left( V-q\right) m^{\prime }(q)+\overline{\varepsilon }}{2\overline{%
\varepsilon }}g(V)dV \\
&&+I(q)\int\nolimits_{q+\frac{\overline{\varepsilon }}{m^{\prime }(q)}%
}^{1}g(V)dV \\
&=&I(q)\left[ \left( G\left( q+\frac{\overline{\varepsilon }}{m^{\prime }(q)}%
\right) -G\left( q-\frac{\overline{\varepsilon }}{m^{\prime }(q)}\right)
\right) +2\int\nolimits_{q}^{q+\frac{\overline{\varepsilon }}{m^{\prime }(q)}%
}\frac{\left( V-q\right) m^{\prime }(q)+\overline{\varepsilon }}{2\overline{%
\varepsilon }}g(V)dV\right]
\end{eqnarray*}%
We used a first-order approximation for $m(q)$, so we can also use one for $%
g(V)$ so that $G(x)=\alpha G(x-V)$ in the neighborhood of $V$. Then $%
g(x)=g(V)$ in the neighborhood of $V$ and we have

\[
E(U\mid q,m^{\prime }(q))=I(q)\left[ \left( G\left( q+\frac{\overline{%
\varepsilon }}{m^{\prime }(q)}\right) -G\left( q-\frac{\overline{\varepsilon 
}}{m^{\prime }(q)}\right) \right) +\frac{1}{2}g(q)dx\right] 
\]

\end{document}
