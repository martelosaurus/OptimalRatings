\documentclass[12pt]{article}
\usepackage[margin=.5in]{geometry}
\usepackage{amsmath}
\usepackage{amsthm}
\usepackage{amsfonts}
\usepackage{amssymb}
\usepackage{mathtools}
\usepackage{float}
\usepackage{graphicx}
\usepackage{setspace}
\usepackage{epstopdf}
\usepackage{showframe}
\usepackage[utf8]{inputenc}
\usepackage[T1]{fontenc}
\usepackage{lmodern}

\newtheorem{assumption}{Assumption}
\newtheorem{lemma}{Lemma}

\DeclareMathOperator*{\argmin}{argmin}
\DeclareMathOperator*{\argmax}{argmax}

\begin{document}
\onehalfspacing
Now consider an alternative utility function. The receiver has a threshold
for quality $V$ so that she receives a payoff of 1 if $q>V$ and $A=1$ or if $%
q<V$ and $A=0$. She receives a payoff of 0 if $q<V$ and $A=1$ or if $q>V$
and $A=0$. The threshold for any given receiver is unknown, and distributed
according to $G$, with density $g$. An alternative interpretation is that $V$
is known, but there are many receivers receiving the same signal, and their
values of $V$ are distributed according to $G$. An additional alternative
interpretation is that $q$ is itself a signal of true quality. $V$ is known,
and there are multiple reviewers, whose reviews are correlated with the
signal $q$. This means that the threshold value of $q$ such that the actor
chooses $A=1$ or $A=0$ is a random variable with distribution $G$. 

Suppose that the optimal response function is $A(\widetilde{m})=1$ if $%
m^{-1}(\widetilde{m})>V$ and $A(\widetilde{m})=0$ if $m^{-1}(\widetilde{m}%
)<V $. We can calculate the expected utility given $q$ as comprised of three
components. First, if $\widetilde{m}<m(V)-\overline{\varepsilon }$, then the
receiver will always take the action $A=0$. If $\widetilde{m}>m(V)+\overline{%
\varepsilon }$, then she will always take action $A=1$. Only if $m\in
\lbrack m(V)-\overline{\varepsilon },m(V)+\overline{\varepsilon }]$ is there
a question of which action she will take.

Because $\varepsilon $ is uniform over $[-\overline{\varepsilon },\overline{%
\varepsilon }]$, and because $m(q)$ is locally linear, the probability that
she takes the correct action is $pr(A=0\mid V>q)=pr(A=1\mid V<q)=\frac{%
\left( V-q\right) m^{\prime }(q)+\overline{\varepsilon }}{2\overline{%
\varepsilon }}$ for small $\varepsilon $ and $V\in $ $[q-\frac{\overline{%
\varepsilon }}{m^{\prime }(q)},q+\frac{\overline{\varepsilon }}{m^{\prime
}(q)}]$. For a given message function $m^{\prime }(q)$, and a given $q$, the
expected utility is: \bigskip

\begin{eqnarray*}
E(U &\mid &q,m^{\prime }(q))=I(q)\int\nolimits_{0}^{q-\frac{\overline{%
\varepsilon }}{m^{\prime }(q)}}g(V)dV \\
&&+I(q)\int\nolimits_{q-\frac{\overline{\varepsilon }}{m^{\prime }(q)}}^{q}%
\frac{\left( q-V\right) m^{\prime }(q)+\overline{\varepsilon }}{2\overline{%
\varepsilon }}g(V)dV \\
&&+I(q)\int\nolimits_{q}^{q+\frac{\overline{\varepsilon }}{m^{\prime }(q)}}%
\frac{\left( V-q\right) m^{\prime }(q)+\overline{\varepsilon }}{2\overline{%
\varepsilon }}g(V)dV \\
&&+I(q)\int\nolimits_{q+\frac{\overline{\varepsilon }}{m^{\prime }(q)}%
}^{1}g(V)dV \\
&=&I(q)\left[ \left( G\left( q+\frac{\overline{\varepsilon }}{m^{\prime }(q)}%
\right) -G\left( q-\frac{\overline{\varepsilon }}{m^{\prime }(q)}\right)
\right) +2\int\nolimits_{q}^{q+\frac{\overline{\varepsilon }}{m^{\prime }(q)}%
}\frac{\left( V-q\right) m^{\prime }(q)+\overline{\varepsilon }}{2\overline{%
\varepsilon }}g(V)dV\right]
\end{eqnarray*}%
We used a first-order approximation for $m(q)$, so we can also use one for $%
g(V)$ so that $G(x)=\alpha G(x-V)$ in the neighborhood of $V$. Then $%
g(x)=g(V)$ in the neighborhood of $V$ and we have

\[
E(U\mid q,m^{\prime }(q))=I(q)\left[ \left( G\left( q+\frac{\overline{%
\varepsilon }}{m^{\prime }(q)}\right) -G\left( q-\frac{\overline{\varepsilon 
}}{m^{\prime }(q)}\right) \right) +\frac{1}{2}g(q)dx\right] 
\]
\end{document}