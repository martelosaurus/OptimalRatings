\documentclass[12pt]{article}
\usepackage{amsmath,amssymb,amsfonts,amsthm,fullpage,setspace,float,graphicx}
\usepackage[round]{natbib}
\newtheorem{corollary}{Corollary}
\newtheorem{lemma}{Lemma}
\newtheorem{definition}{Definition}
\newtheorem{proposition}{Proposition}
\renewcommand{\qedsymbol}{$\blacksquare$}
\DeclareMathOperator*{\Beta}{Beta}
\DeclareMathOperator*{\argmin}{argmin}
\DeclareMathOperator*{\argmax}{argmax}
\begin{document}
\noindent\textit{I've done the math enough to know the dangers of our second guessing\\ Doomed to crumble unless we grow, and strengthen our communication}\\ \\$\sim$ From ``Schism,'' by Tool. \\
%\noindent\textit{Let your `Yes' be `Yes,' and your `No,' `No.' For whatever is more than these is from the evil one.}

%%%%%%%%%%%%%%%
%%% SUMMARY %%%
%%%%%%%%%%%%%%%
\noindent\underline{Summary}. This note establishes two facts about the cooperative communication game under uniform noise, quadratic loss, and bounded messages. First, there does not exist an equilibrium in which the message function is continuous and strictly increasing. Second, there \textit{does} exist an equilibrium in which the message function is piecewise constant. I have not ruled out equilibria in which the message function is constant in some neighborhoods, and strictly increasing in others. Unless otherwise noted, I follow the conventions of the original paper.\\

%%%%%%%%%%%%%%%%%%%%%%%%%%%%%%%%%%%%%%%%
%%% UNIFORM NOISE AND QUADRATIC LOSS %%%
%%%%%%%%%%%%%%%%%%%%%%%%%%%%%%%%%%%%%%%%
\noindent\underline{Uniform Noise and Quadratic Loss}. Set $L(x)=\tfrac{1}{2}x^2$ and $g(q)=\mathbf{1}_{0\leq q\leq1}$. Fix $\bar{\epsilon}>0$, and $I\in C[0,1]$, where $I>0$ on $[0,1]$. Action functions are maps from \textit{received messages} to actions: $\tilde{m}\mapsto a(\tilde{m})$. Message functions are maps from states to \textit{sent messages}: $q\mapsto m(q)$. Sent messages may be no smaller than $\underline{m}$ and no larger than $\overline{m}$, where $\overline{m}-\underline{m}>2\bar{\epsilon}$. Sent and received messages are related by the identity $\tilde{m}=m(q)+e$, where $e$ is uniformly distributed on $[-\bar{\epsilon},\bar{\epsilon}]$. $q$ is known to the sender, but not the receiver; $e$ is known to neither the sender, nor the receiver. 

%%%%%%%%%%%%%%%%%%%%%%%%%%%%%%%
%%% DEFINITION: ACTION RULE %%%
%%%%%%%%%%%%%%%%%%%%%%%%%%%%%%%
\begin{definition}
A map $a:[\underline{m}-\bar{\epsilon},\overline{m}+\bar{\epsilon}]\rightarrow\mathbb{R}$ is an action function if it is weakly increasing.  
\end{definition}

%%%%%%%%%%%%%%%%%%%%%%%%%%%%%%%%
%%% DEFINITION: MESSAGE RULE %%%
%%%%%%%%%%%%%%%%%%%%%%%%%%%%%%%%
\noindent Message functions are similarly defined, but with the additional requirement that they satisfy certain boundary conditions.
\begin{definition}
A map $m:[0,1]\rightarrow[\underline{m},\overline{m}]$ is a message function if it is weakly increasing, and if $m(0)=\underline{m}$ and $m(1)=\overline{m}$. 
\end{definition}
\noindent The boundary conditions $m(0)=\underline{m}$ and $m(1)=\overline{m}$ are made to quarantee the uniqueness of an equilibrium. Note that the receiver's action has domain $[\underline{m}-\bar{\epsilon},\overline{m}+\bar{\epsilon}]$. The sender's message has codomain $[\underline{m},\overline{m}]$, but it is corrupted by noise. Although a \textit{sent} message may be no smaller than $\underline{m}$, and no larger than $\overline{m}$, a \textit{received} message may be as small as $\underline{m}-\bar{\epsilon}$, or as large as $\overline{m}+\bar{\epsilon}$. The receiver must choose an action in $[\underline{m},\overline{m}]$ for messages less than $\underline{m}$ or greater than $\overline{m}$. Since $e$ is uniformly distributed on $[-\bar{\epsilon},\bar{\epsilon}]$, $\tilde{m}$ has density 
\begin{equation}
f(\tilde{m})=\frac{1}{2\bar{\epsilon}}\mathbf{1}_{m(q)-\bar{\epsilon}\leq e\leq m(q)+\bar{\epsilon}}.
\end{equation}
The sender and receiver minimize the objectives
\begin{align}
S(a,m;q)&=\int_{m(q)-\bar{\epsilon}}^{m(q)+\bar{\epsilon}}{I(q)L(a(\tilde{m})-q)f(\tilde{m})d\tilde{m}}\text{, and}\\
R(a,m;\tilde{m})&=\int_{0}^{1}{I(q)L(a(\tilde{m})-q)g(q|\tilde{m})dq}
\end{align}
respectively. Define
\begin{align}
\overline{Q}(\tilde{m})&:=\{q\in[0,1]\:|\:m(q)-\bar{\epsilon}\leq\tilde{m}\}\\
\underline{Q}(\tilde{m})&:=\{q\in[0,1]\:|\:\tilde{m}\leq m(q)-\bar{\epsilon}\}\\
Q(\tilde{m})&:=\{q\in[0,1]\:|\:m(q)-\bar{\epsilon}\leq\tilde{m}\leq m(q)+\bar{\epsilon}\}.
\end{align}
Define $\overline{q}(\tilde{m}):=\sup Q(\tilde{m})$ and $\underline{q}(\tilde{m}):=\inf Q(\tilde{m})$. Given a received message $\tilde{m}$, $\overline{q}(\tilde{m})$ is the least upper bound of states that could have resulted in $\tilde{m}$, while $\underline{q}(\tilde{m})$ is the largest lower bound of states that could have resulted in $\tilde{m}$.

%%%%%%%%%%%%%%%%%%% 
%%% STATE LEMMA %%%
%%%%%%%%%%%%%%%%%%%
\begin{lemma}[Properties of $\underline{q}$ and $\overline{q}$]\label{lemma:props}
$\underline{q}$ and $\overline{q}$ are increasing, $\overline{q}>\underline{q}$ on $(\underline{m}-\bar{\epsilon},\overline{m}+\bar{\epsilon})$, $\overline{q}(\underline{m}-\bar{\epsilon})=\overline{q}(\overline{m}+\bar{\epsilon})$, and $\overline{q}(\underline{m}-\bar{\epsilon})=\overline{q}(\overline{m}+\bar{\epsilon})$.
\end{lemma}
\begin{proof}
Choose $\tilde{x},\tilde{y}\in[\underline{m}-\bar{\epsilon},\overline{m}+\bar{\epsilon}]$ with $\tilde{x}<\tilde{y}$. $\overline{Q}(\tilde{x})\subseteq\overline{Q}(\tilde{y})$, and hence 
\begin{equation}
\overline{q}(\tilde{x})=\sup\overline{Q}(\tilde{x})\leq\sup\overline{Q}(\tilde{y})=\overline{q}(\tilde{y}).
\end{equation} 
Similarly for $\underline{q}$. Now $\underline{Q}(\tilde{x})\subset\overline{Q}(\tilde{x})$, and hence
\begin{equation}
\underline{q}(\tilde{x})=\inf\underline{Q}(\tilde{x})\leq\sup\overline{Q}(\tilde{x})=\overline{q}(\tilde{x}).
\end{equation}
The last properties follow by direct observation.  
\end{proof}

\noindent If $\tilde{m}\in(\underline{m}-\bar{\epsilon},\overline{m}+\bar{\epsilon})$, then the receiver has beliefs, 
\begin{equation}
g(q|\tilde{m})=\frac{f(\tilde{m}|q)g(q)}{f(\tilde{m})}=\frac{\frac{1}{2\bar{\epsilon}}\mathbf{1}_{m(q)-\bar{\epsilon}\leq\tilde{m}\leq m(q)+\bar{\epsilon}}\cdot\mathbf{1}_{0\leq q\leq 1}}{\int_{-\infty}^{\infty}{\frac{1}{2\bar{\epsilon}}\mathbf{1}_{m(t)-\bar{\epsilon}\leq\tilde{m}\leq m(t)+\bar{\epsilon}}\cdot\mathbf{1}_{0\leq t\leq 1}}dt}=\frac{\mathbf{1}_{\underline{q}(\tilde{m})\leq q\leq\overline{q}(\tilde{m})}}{\overline{q}(\tilde{m})-\underline{q}(\tilde{m})}.
\end{equation}
If she receives the message $\underline{m}-\bar{\epsilon}$, then she believes that the state is $0$ almost surely, while if she receives the message $\overline{m}+\bar{\epsilon}$, then she believes that the state is $1$ almost surely. Formally, $g(q|\underline{m}-\bar{\epsilon})=\delta(q-(\underline{m}-\bar{\epsilon}))$ and $g(q|\overline{m}+\bar{\epsilon})=\delta(q-(\overline{m}+\bar{\epsilon}))$, where $\delta(\bullet)$ is the Dirac delta function.

%%%%%%%%%%%%%%%%%%%%%%%%%%%%%%%%
%%% RECEIVER'S BEST RESPONSE %%%
%%%%%%%%%%%%%%%%%%%%%%%%%%%%%%%%
\begin{lemma}[Receiver's Best Response]\label{lemma:receiver}
For $\tilde{m}\in(\underline{m}-\bar{\epsilon},\overline{m}+\bar{\epsilon})$, the receiver's best response is given by
\begin{equation}
a(\tilde{m})=\frac{\int_{\underline{q}(\tilde{m})}^{\overline{q}(\tilde{m})}{rI(r)dr}}{\int_{\underline{q}(\tilde{m})}^{\overline{q}(\tilde{m})}{I(r)dr}}.
\end{equation}
Moreover, $a(\underline{m}-\bar{\epsilon})=0$ and $a(\overline{m}+\bar{\epsilon})=1$. 
\end{lemma}
\begin{proof}
For $\tilde{m}\in(\underline{m}-\bar{\epsilon},\overline{m}+\bar{\epsilon})$, $R$ is strictly convex in its first argument
\begin{equation}
R_{aa}(a,m,\tilde{m})=\int_{\underline{q}(\tilde{m})}^{\overline{q}(\tilde{m})}{\frac{I(q)L''(a-q)dq}{\overline{q}(\tilde{m})-\underline{q}(\tilde{m})}}=\int_{\underline{q}(\tilde{m})}^{\overline{q}(\tilde{m})}{\frac{I(q)dq}{\overline{q}(\tilde{m})-\underline{q}(\tilde{m})}}>0.
\end{equation}
The solution of the first-order condition is the unique minimizer:
\begin{equation}
0=R_{a}(\tilde{m},a,m)=\int_{\underline{q}(\tilde{m})}^{\overline{q}(\tilde{m})}{\frac{I(q)L'(a-q)dq}{\overline{q}(\tilde{m})-\underline{q}(\tilde{m})}}=\int_{\underline{q}(\tilde{m})}^{\overline{q}(\tilde{m})}{\frac{I(q)(a-q)dq}{\overline{q}(\tilde{m})-\underline{q}(\tilde{m})}}.
\end{equation}
from which the formula obtains. If she receives the message $\tilde{m}-\bar{\epsilon}$, then she beliefs that the state is $0$ almost surely, in which case she plays $a(\tilde{m}-\bar{\epsilon})=0$. If she receives the message $\tilde{m}+\bar{\epsilon}$, then she beliefs that the state is $1$ almost surely, in which case she plays $a(\tilde{m}+\bar{\epsilon})=1$. It is evident that $a([\underline{m}-\bar{\epsilon},\overline{m}+\bar{\epsilon}])\subseteq[0,1]$. Choose two received messages, $\tilde{x},\tilde{y}\in(\underline{m}-\bar{\epsilon},\overline{m}+\bar{\epsilon})$ with $\tilde{x}<\tilde{y}$. Since $\overline{q}$ and $\underline{q}$ are weakly increasing, $\overline{q}(\tilde{y})\geq\overline{q}(\tilde{x})$ and $\underline{q}(\tilde{y})\geq\underline{q}(\tilde{x})$, and hence
\begin{equation}
a(\tilde{y})-a(\tilde{x})=\frac{\int_{\underline{q}(\tilde{y})}^{\overline{q}(\tilde{y})}{rI(r)dr}}{\int_{\underline{q}(\tilde{y})}^{\overline{q}(\tilde{y})}{I(r)dr}}-\frac{\int_{\underline{q}(\tilde{x})}^{\overline{q}(\tilde{x})}{rI(r)dr}}{\int_{\underline{q}(\tilde{x})}^{\overline{q}(\tilde{x})}{I(r)dr}}=\frac{\int_{\underline{q}(\tilde{y})}^{\overline{q}(\tilde{y})}{\int_{\underline{q}(\tilde{x})}^{\overline{q}(\tilde{x})}{(s-t)I(s)I(t)dsdt}}}{\int_{\underline{q}(\tilde{y})}^{\overline{q}(\tilde{y})}{\int_{\underline{q}(\tilde{x})}^{\overline{q}(\tilde{x})}{I(s)I(t)dsdt}}}\geq0.
\end{equation}
Hence, $a$ is weakly increasing. We conclude that it is an action function.  
\end{proof}

%%%%%%%%%%%%%%%%%%%%%%%%%%%%%%%%%%%%%%%%%%%
%%% PIECEWISE CONSTANT MESSAGE FUNCTION %%%
%%%%%%%%%%%%%%%%%%%%%%%%%%%%%%%%%%%%%%%%%%%
\begin{definition}[Piecewise Constant Message Function]
A message function is said to be piecewise constant if there is an integer $N$, a paritition $M=\{m_{0},m_{1},\cdots,m_{n}\}$ of $[\underline{m},\overline{m}]$, and a partition $Q=\{q_{0},q_{1},\cdots,q_{n}\}$ of $[0,1]$ such that
\begin{equation}
m(q)=\sum_{n=0}^{N-1}{m_{n}\mathbf{1}_{q_{n}\leq q<q_{n+1}}}
\end{equation}
for $q>0$ and $m(1)=\overline{m}$.
\end{definition}

%%%%%%%%%%%%%%%%%%%%%%%%
%%% MAIN PROPOSITION %%%
%%%%%%%%%%%%%%%%%%%%%%%%
\begin{proposition}
The unique equilbirum of the cooperative communication game has a piecewise constant message function. 
\end{proposition}
\begin{proof}
Define $\bar{a}(m)=\tfrac{1}{2}(a(m+\bar{\epsilon})+a(m-\bar{\epsilon}))$,
\begin{align}
r_{1}(m;q)&:=a(m+\bar{\epsilon})-a(m-\bar{\epsilon}),\text{and }\\
r_{2}(m;q)&:=\bar{a}(m)-q.
\end{align}
Since $a$ is weakly increasing, $r_{1}$ is weakly positive and $r_{2}$ is weakly increasing. Observe that
\begin{align}
S_{m(q)}(a,m;q)&=L(a(m+\bar{\epsilon})-q)-L(a(m-\bar{\epsilon})-q)\\
&=\tfrac{1}{2}(a(m+\bar{\epsilon})-q)^{2}-\tfrac{1}{2}(a(m-\bar{\epsilon})-q)^{2}\\
&=\tfrac{1}{2}(a(m+\bar{\epsilon})-a(m-\bar{\epsilon}))(a(m+\bar{\epsilon})+a(m-\bar{\epsilon})-2q)\\
&=(a(m+\bar{\epsilon})-a(m-\bar{\epsilon}))(\bar{a}(m)-q)\\
&=r_{1}(m)r_{2}(m).
\end{align}
First, consider the left boundary. Fix $q\in[0,\bar{a}(\underline{m}))$. There are two cases: either
\begin{enumerate}
\item $a(\underline{m}+\bar{\epsilon})=a(\underline{m}-\bar{\epsilon})=0$, in which case $\overline{q}(\underline{m}+\bar{\epsilon})=\underline{q}(\underline{m}+\bar{\epsilon})=0$ (Lemma \ref{lemma:receiver}), which is a contradiction (Lemma \ref{lemma:props}); or
\item $a(\underline{m}+\bar{\epsilon})>a(\underline{m}-\bar{\epsilon})$, in which case $S_{m}(a,\underline{m};q)>0$, and she finds $m^{*}(q)=\underline{m}$ optimal.
\end{enumerate} 
Next, consider the right boundary. Fix $q\in(\bar{a}(\overline{m}),1]$. There are two cases: either
\begin{enumerate}
\item $a(\overline{m}+\bar{\epsilon})=a(\overline{m}-\bar{\epsilon})=1$, in which case $\overline{q}(\overline{m}+\bar{\epsilon})=\underline{q}(\overline{m}+\bar{\epsilon})=1$ (Lemma \ref{lemma:receiver}), which is a contradiction (Lemma \ref{lemma:props}); or
\item $a(\overline{m}+\bar{\epsilon})>a(\overline{m}-\bar{\epsilon})$, in which case $S_{m}(a,\overline{m};q)>0$, and she finds $m^{*}(q)=\overline{m}$ optimal.
\end{enumerate}
Finally, consider the mid-section. Choose $q\in[E[a(\underline{m}+e)],E[a(\overline{m}+e)]]$. Now $r_{2}(\underline{m};q)<0$ and $r_{2}(\overline{m};q)>0$, and hence $S_{m}(a,\underline{m};q)<0$ and $S_{m}(a,\overline{m};q)>0$. Define
\begin{align}
\hat{m}&:=\sup\{m\in[\underline{m},\overline{m}]\:|\:S_{m}(a,m;q)\leq0\}\\
\check{m}&:=\inf\{m\in[\underline{m},\overline{m}]\:|\:S_{m}(a,m;q)\geq0\}.
\end{align}
She finds $m^{*}(q)=\hat{m}=\check{m}$ optimal.
\end{proof}

\end{document}