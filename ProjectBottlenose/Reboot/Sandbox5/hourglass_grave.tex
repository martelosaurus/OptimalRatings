\documentclass[12pt]{article}
\usepackage{amsmath,amssymb,amsfonts,amsthm,fullpage,setspace,float,graphicx}
\usepackage[round]{natbib}
\newtheorem{corollary}{Corollary}
\newtheorem{lemma}{Lemma}
\newtheorem{definition}{Definition}
\newtheorem{proposition}{Proposition}
\renewcommand{\qedsymbol}{$\blacksquare$}
\DeclareMathOperator*{\Beta}{Beta}
\DeclareMathOperator*{\argmin}{argmin}
\DeclareMathOperator*{\argmax}{argmax}
\begin{document}

%%%%%%%%%%%%%%%%%%%%%%%%%%%%%%%%%%%%%%%%
%%% UNIFORM NOISE AND QUADRATIC LOSS %%%
%%%%%%%%%%%%%%%%%%%%%%%%%%%%%%%%%%%%%%%%
\noindent Set $L(x)=\tfrac{1}{2}x^2$ and $g(q)=\mathbf{1}_{0\leq q\leq1}$. Fix $\bar{\epsilon}>0$, and $I\in C[0,1]$, where $I>0$ on $[0,1]$. Action functions are maps from \textit{received messages} to actions: $\tilde{m}\mapsto a(\tilde{m})$. Message functions are maps from states to \textit{sent messages}: $q\mapsto m(q)$. Sent messages may be no smaller than $\underline{m}$ and no larger than $\overline{m}$, where $\overline{m}-\underline{m}>2\bar{\epsilon}$. Sent and received messages are related by the identity $\tilde{m}=m(q)+e$, where $e$ is uniformly distributed on $[-\bar{\epsilon},\bar{\epsilon}]$. $q$ is known to the sender, but not the receiver; $e$ is known to neither the sender, nor the receiver. 

%%%%%%%%%%%%%%%%%%%%%%%%%%%%%%%
%%% DEFINITION: ACTION RULE %%%
%%%%%%%%%%%%%%%%%%%%%%%%%%%%%%%
\begin{definition}
A map $a:[\underline{m}-\bar{\epsilon},\overline{m}+\bar{\epsilon}]\rightarrow\mathbb{R}$ is an action function if it is weakly increasing.  
\end{definition}

%%%%%%%%%%%%%%%%%%%%%%%%%%%%%%%%
%%% DEFINITION: MESSAGE RULE %%%
%%%%%%%%%%%%%%%%%%%%%%%%%%%%%%%%
\noindent Message functions are similarly defined, but with the additional requirement that they satisfy certain boundary conditions.
\begin{definition}
A map $m:[0,1]\rightarrow[\underline{m},\overline{m}]$ is a message function if it is weakly increasing, and if $m(0)=\underline{m}$ and $m(1)=\overline{m}$. 
\end{definition}
\noindent The boundary conditions $m(0)=\underline{m}$ and $m(1)=\overline{m}$ are made to quarantee the uniqueness of an equilibrium. Note that the receiver's action has domain $[\underline{m}-\bar{\epsilon},\overline{m}+\bar{\epsilon}]$. The sender's message has codomain $[\underline{m},\overline{m}]$, but it is corrupted by noise. Although a \textit{sent} message may be no smaller than $\underline{m}$, and no larger than $\overline{m}$, a \textit{received} message may be as small as $\underline{m}-\bar{\epsilon}$, or as large as $\overline{m}+\bar{\epsilon}$. The receiver must choose an action in $\mathbb{R}$ for messages less than $\underline{m}$ or greater than $\overline{m}$. Since $e$ is uniformly distributed on $[-\bar{\epsilon},\bar{\epsilon}]$, $\tilde{m}$ has density 
\begin{equation}
f(\tilde{m})=\frac{1}{2\bar{\epsilon}}\mathbf{1}_{m(q)-\bar{\epsilon}\leq\tilde{m}\leq m(q)+\bar{\epsilon}}.
\end{equation}
The sender's and receiver's payoffs are 
\begin{align}
S(a,m;q)&=\int_{m(q)-\bar{\epsilon}}^{m(q)+\bar{\epsilon}}{I(q)L(a(\tilde{m})-q)f(\tilde{m})d\tilde{m}}\text{, and}\\
R(a,m;\tilde{m})&=\int_{0}^{1}{I(q)L(a(\tilde{m})-q)g(q|\tilde{m})dq}
\end{align}
respectively. 

%%%%%%%%%%%%%%%%%%%%%%%%%%%%%%%%%%%%%%%%%%%%%%%%%%%%%%%%%%%%%
%%% CONTINUOUS AND STRICTLY INCREASING MESSAGE FUNCTIONS %%%%
%%%%%%%%%%%%%%%%%%%%%%%%%%%%%%%%%%%%%%%%%%%%%%%%%%%%%%%%%%%%%
\begin{proposition}[Continuous and Strictly Increasing Message Functions]
There does not exist an equilibrium in which the message function is continuous and strictly increasing, and the receiver believes it to be so.
\end{proposition}
\begin{proof}
By way of contradiction, let $m$ be an equilibrium message function, and suppose that it is continuous and strictly increasing. Put $\ell=m^{-1}$. Since $m$ is continuous and strictly increasing, so too is $\ell$. Moreover, $\ell(\underline{m})=0$ and $\ell(\overline{m})=1$. Define 
\begin{align}
\overline{q}(\tilde{m})&=\ell(\min\{\tilde{m}+\bar{\epsilon},\overline{m}\}),\text{ and}\\
\underline{q}(\tilde{m})&=\ell(\max\{\tilde{m}-\bar{\epsilon},\underline{m}\}).
\end{align} 
Given a received message $\tilde{m}$, $\overline{q}(\tilde{m})$ is the \textit{largest} possible state that could have resulted in $\tilde{m}$, while $\underline{q}(\tilde{m})$ is the \textit{smallest} possible state that could have resulted in $\tilde{m}$. For $\tilde{m}\in(\underline{m}-\bar{\epsilon},\overline{m}+\bar{\epsilon})$, $\overline{q}(\tilde{m})>\underline{q}(\tilde{m})$. Moreover, $\overline{q}(\underline{m}-\bar{\epsilon})=\underline{q}(\underline{m}-\bar{\epsilon})$, and $\overline{q}(\underline{m}+\bar{\epsilon})=\underline{q}(\underline{m}+\bar{\epsilon})$. As they will appear often, define 
\begin{align}
\underline{\mathcal{I}}&:=(\underline{m}-\bar{\epsilon},\underline{m}+\bar{\epsilon})\text{,}\\
\mathcal{I}&:=(\underline{m}+\bar{\epsilon},\overline{m}-\bar{\epsilon})\text{, and}\\
\overline{\mathcal{I}}&:=(\overline{m}-\bar{\epsilon},\overline{m}+\bar{\epsilon}).
\end{align}
On $\underline{\mathcal{I}}$, $\underline{q}'=0$ and $\overline{q}'>0$; on $\mathcal{I}$, $\underline{q}'>0$ and $\overline{q}'>0$; on $\overline{\mathcal{I}}$, $\underline{q}'>0$ and $\overline{q}'=0$. If $\tilde{m}\in(\underline{m}-\bar{\epsilon},\overline{m}+\bar{\epsilon})$, then the receiver has beliefs, 
\begin{equation}
g(q|\tilde{m})=\frac{f(\tilde{m}|q)g(q)}{f(\tilde{m})}=\frac{\frac{1}{2\bar{\epsilon}}\mathbf{1}_{m(q)-\bar{\epsilon}\leq\tilde{m}\leq m(q)+\bar{\epsilon}}\cdot\mathbf{1}_{0\leq q\leq 1}}{\int_{-\infty}^{\infty}{\frac{1}{2\bar{\epsilon}}\mathbf{1}_{m(t)-\bar{\epsilon}\leq\tilde{m}\leq m(t)+\bar{\epsilon}}\cdot\mathbf{1}_{0\leq t\leq 1}}dt}=\frac{\mathbf{1}_{\underline{q}(\tilde{m})\leq q\leq\overline{q}(\tilde{m})}}{\overline{q}(\tilde{m})-\underline{q}(\tilde{m})}.
\end{equation}
If she receives the message $\underline{m}-\bar{\epsilon}$, then she believes that the state is $0$ almost surely, while if she receives the message $\overline{m}+\bar{\epsilon}$, then she believes that the state is $1$ almost surely. Formally, $g(q|\underline{m}-\bar{\epsilon})=\delta(q-(\underline{m}-\bar{\epsilon}))$ and $g(q|\overline{m}+\bar{\epsilon})=\delta(q-(\overline{m}+\bar{\epsilon}))$, where $\delta(\bullet)$ is the Dirac delta function. Next, we compute the receiver's best response action function. If $\tilde{m}\in(\underline{m}-\bar{\epsilon},\overline{m}+\bar{\epsilon})$, then $R$ is strictly convex in its first argument:
\begin{equation}
R_{aa}(a,m;\tilde{m})=\int_{\underline{q}(\tilde{m})}^{\overline{q}(\tilde{m})}{\frac{I(q)L''(a(\tilde{m})-q)dq}{\overline{q}(\tilde{m})-\underline{q}(\tilde{m})}}=\int_{\underline{q}(\tilde{m})}^{\overline{q}(\tilde{m})}{\frac{I(q)dq}{\overline{q}(\tilde{m})-\underline{q}(\tilde{m})}}>0.
\end{equation}
The solution of the first-order condition is the unique minimizer:
\begin{equation}
0=R_{a}(\tilde{m},a,m)=\int_{\underline{q}(\tilde{m})}^{\overline{q}(\tilde{m})}{\frac{I(q)L'(a(\tilde{m})-q)dq}{\overline{q}(\tilde{m})-\underline{q}(\tilde{m})}}=\int_{\underline{q}(\tilde{m})}^{\overline{q}(\tilde{m})}{\frac{I(q)(a(\tilde{m})-q)dq}{\overline{q}(\tilde{m})-\underline{q}(\tilde{m})}},
\end{equation}
which implies that the receiver plays
\begin{equation}
a(\tilde{m})=\left[\int_{\underline{q}(\tilde{m})}^{\overline{q}(\tilde{m})}{I(r)dr}\right]^{-1}\left[\int_{\underline{q}(\tilde{m})}^{\overline{q}(\tilde{m})}{rI(r)dr}\right].
\end{equation}
If she receives the message $\tilde{m}-\bar{\epsilon}$, then she believes that the state is $0$ almost surely, in which case she plays $a(\tilde{m}-\bar{\epsilon})=0$. If she receives the message $\tilde{m}+\bar{\epsilon}$, then she beliefs that the state is $1$ almost surely, in which case she plays $a(\tilde{m}+\bar{\epsilon})=1$. Since $\underline{q}$ and $\overline{q}$ are continuous, so too is $a$. On $\underline{\mathcal{I}}\cup\mathcal{I}\cup\overline{\mathcal{I}}$, 
\begin{align}
a'&=\left[\int_{\underline{q}}^{\overline{q}}{I(r)dr}\right]^{-2}\left[(\overline{q}'\overline{q}I(\overline{q})-\underline{q}'\underline{q}I(\underline{q}))\int_{\underline{q}}^{\overline{q}}{I(r)dr}-(\overline{q}'I(\overline{q})-\underline{q}'I(\underline{q}))\int_{\underline{q}}^{\overline{q}}{rI(r)dr}\right]\\
&=\left[\int_{\underline{q}}^{\overline{q}}{I(r)dr}\right]^{-2}\left[\overline{q}'I(\overline{q})\int_{\underline{q}}^{\overline{q}}{(\overline{q}-r)dr}+\underline{q}'I(\underline{q})\int_{\underline{q}}^{\overline{q}}{(r-\underline{q})dr}\right]>0.
\end{align}
$a$ is strictly increasing on $\underline{\mathcal{I}}\cup\mathcal{I}\cup\overline{\mathcal{I}}$, and  since it is continuous on $[\underline{m}-\bar{\epsilon},\overline{m}+\bar{\epsilon}]$, it is strictly increasing on $[\underline{m}-\bar{\epsilon},\overline{m}+\bar{\epsilon}]$. Finally, we show that the sender has a profitable deviation. Define $\bar{a}(m):=\tfrac{1}{2}(a(m+\bar{\epsilon})+a(m-\bar{\epsilon}))$. Fix $q\in[0,\bar{a}(\underline{m}))$ and define
\begin{align}
r_{1}(m)&:=a(m+\bar{\epsilon})-a(m-\bar{\epsilon})\text{, and}\\
r_{2}(m)&:=\bar{a}(m)-q.
\end{align}
Note that $r_{2}(\underline{m})>0$. Since $a$ is strictly increasing, $r_{1}$ is strictly positive, and $r_{2}$ is strictly increasing. Observe that
\begin{align}
S_{m(q)}(a,m;q)&=L(a(m+\bar{\epsilon})-q)-L(a(m-\bar{\epsilon})-q)\\
&=\tfrac{1}{2}(a(m+\bar{\epsilon})-q)^{2}-\tfrac{1}{2}(a(m-\bar{\epsilon})-q)^{2}\\
&=\tfrac{1}{2}(a(m+\bar{\epsilon})-a(m-\bar{\epsilon}))(a(m+\bar{\epsilon})+a(m-\bar{\epsilon})-2q)\\
&=(a(m+\bar{\epsilon})-a(m-\bar{\epsilon}))(\tfrac{1}{2}(a(m+\bar{\epsilon})+a(m-\bar{\epsilon}))-q)\\
&=r_{1}(m)r_{2}(m).
\end{align}
Since $r_{2}(\underline{m})>0$ and $r_{2}$ is strictly increasing, $r_{2}>0$ on $[\underline{m},\overline{m}]$, and hence $S_{m(q)}>0$ on $[\underline{m},\overline{m}]$. The sender finds $m^{*}(q)=\underline{m}$ optimal. The message function obtained cannot be the equilibrium message function, since $m$ is strictly increasing, and $m^{*}$ is not. Therefore, $m^{*}$ is a profitable deviation for the sender, contradicting the assumption that $m$ is an equilibrium message function. 
\end{proof}
The above proof suggests that a continuous and strictly increasing message function could be rectified by taking $\overline{m}\rightarrow\infty$ and $\underline{m}\rightarrow-\infty$. This is not the case. One obtains $\ell(q)=\tfrac{1}{2}$ for $q\in(0,1)$, which is also constant. 

\end{document}