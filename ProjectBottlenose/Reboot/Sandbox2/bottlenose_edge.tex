\documentclass[12pt]{article}
\usepackage{amsmath,amssymb,amsfonts,amsthm,fullpage,setspace,float,pdflscape,graphicx}
\usepackage[round]{natbib}
\newtheorem{corollary}{Corollary}
\newtheorem{lemma}{Lemma}
\newtheorem{definition}{Definition}
\newtheorem{proposition}{Proposition}
\renewcommand{\qedsymbol}{$\blacksquare$}
\begin{document}
\begin{landscape}
\noindent\textit{I've done the math enough to know the dangers of our second guessing.\\ Doomed to crumble unless we grow, and strengthen our communication.}\\ \\$\sim$ From ``Schism,'' by Tool. \\ \\
\noindent\textit{Uniform Noise, Quadratic Loss.} Set $L(x)=\frac{1}{2}x^2$. Fix $\bar{\epsilon}>0$, and $I\in C^{\infty}(\mathbb{R})$. Let $q\in\mathbb{R}$ be the state of the world. As they will appear frequently, define $\mathcal{I}^{0}(s,t):=\int_{s}^{t}{I(r)dr}$, and $\mathcal{I}^{1}(s,t):=\int_{s}^{t}{rI(r)dr}$. Action rules are maps from \textit{received messages} to actions: $\tilde{m}\mapsto a(\tilde{m})$. Message rules are maps from states to \textit{sent messages}: $q\mapsto m(q)$. Sent and received messages are related by the identity $\tilde{m}=m(q)+e$, where $e\sim\mathcal{N}(0,1)$. Let $f$ denote the density of $e$. $q$ is known to the sender, but not the receiver; $e$ is known to neither the sender, nor the receiver. Action and message rules are said to be \textit{admissable} if they are diffeomorphisms (differentiable bijection). An immediate consequence is that action and messages rules are strictly monotone. Without loss of generality, we take them to be strictly increasing. Throughtout, will will find it more convenient to work with the inverse message rule. Put $\ell=m^{-1}$. Note that $\ell$ is a diffeomorphism iff $m$ is a diffeomorphism. We will take as the ambient solution space the Sobelev space $W^{1,2}(\mathbb{R})$, with norm
\begin{equation} 
||f||=\sqrt{\int_{-\infty}^{\infty}{f(x)^{2}dx}+\int_{-\infty}^{\infty}{f'(x)^{2}dx}}.
\end{equation}
\noindent\textit{Notation:} As shorthand, let the subscripts ``$+$'' and ``$-$'' denote right and left translation by $\bar{\epsilon}$ respectively: for each $f:\mathbb{R}\rightarrow[0,1]$, let $f_{+\bar{\epsilon}},f_{-\bar{\epsilon}}:\mathbb{R}\rightarrow[0,1]$ be given by $f_{+\bar{\epsilon}}(x)=f(x+\bar{\epsilon})$ and $f_{-\bar{\epsilon}}(x)=f(x-\bar{\epsilon})$. \\

%%%%%%%%%%%%%%%
%%% BELIEFS %%%
%%%%%%%%%%%%%%%
\noindent\textit{Beliefs.} Let $g(q|\tilde{m})$ denote the receiver's posterior having received the message $\tilde{m}$. $\ell_{+\bar{\epsilon}}(\tilde{m}$ is the \textit{largest} possible state that could have resulted in $\tilde{m}$, while $\ell_{-\bar{\epsilon}}(\tilde{m})$ is the \textit{smallest} possible state that could have resulted in $\tilde{m}$. According to Baye's Law,
\begin{equation}
g(q|\tilde{m})=m'(q)\phi(m(q);\tilde{m},1);
\end{equation}
Observes that
\begin{equation}
\frac{d}{d\tilde{m}}g(q|\tilde{m})=(m(q)-\tilde{m})g(q|\tilde{m}).
\end{equation}

%%%%%%%%%%%%%%%
%%% PAYOFFS %%%
%%%%%%%%%%%%%%%
\noindent\textit{Payoffs.} The sender's and receiver's payoffs are 
\begin{align}
S(a,m;q)&=\int_{-\infty}^{\infty}{I(q)L(a(m(q)+e)-q)f(e)de}\\
R(a,m;\tilde{m})&=\int_{-\infty}^{\infty}{I(q)L(a(\tilde{m})-q)g(q|\tilde{m})dq}
\end{align}
respectively.

%%%%%%%%%%%%%%%%%%%%%%%%%%%%
%%% THE SENDER'S PROBLEM %%% 
%%%%%%%%%%%%%%%%%%%%%%%%%%%%
\begin{lemma}\label{lemma:message}
For each action rule $a$, the sender's best-response message rule is given by
\begin{equation}
\ell(\bar{m})=\frac{a(\bar{m}+\bar{\epsilon})+a(\bar{m}-\bar{\epsilon})}{2}.
\end{equation}
\end{lemma}
\begin{proof}
The sender's problem is to
\begin{equation}
\min_{m(q)\in\mathbb{R}}\int_{-\infty}^{\infty}{I(q)L(a(m(q)+e)-q)f(e)de}
\end{equation}
the first-order condition of which is
\begin{align}
0=\mathcal{S}_{m}(a,m;q)&=\int_{-\infty}^{\infty}{L'(a(m(q)+e)-q)a'(m(q)+e)f(e)de}\\
&=\int_{-\infty}^{\infty}{(a(m(q)+e)-q)a'(m(q)+e)f(e)de}
\end{align}
Suppose that $\mathcal{S}_{m}(a,m;q)=0$. Since $L'$ is odd, 
\begin{align}
\mathcal{S}_{mm}(a,m;q)&=\int_{-\infty}^{\infty}{\{a'(m(q)+e)^2+(a(m(q)+e)-q)a''(m(q)+e)\}f(e)de}\\
&=L'(a(m(q)+\bar{\epsilon})-q)(a'(m(q)+\bar{\epsilon})+a'(m(q)-\bar{\epsilon}))>0.
\end{align}
\end{proof}

%%%%%%%%%%%%%%%%%%%%%%%%%%%%%%
%%% THE RECEIVER'S PROBLEM %%%
%%%%%%%%%%%%%%%%%%%%%%%%%%%%%%
\begin{lemma}
For each message rule $\ell$, the receiver's best-response action rule is given by
\begin{equation}
a(\tilde{m})=\frac{\int_{-\infty}^{\infty}{qI(q)g(q|\tilde{m})dq}}{\int_{-\infty}^{\infty}{I(q)g(q|\tilde{m})dq}}
\end{equation}
\end{lemma}
\begin{proof}
The receiver's problem is to choose an action rule $a$ such that for each $\tilde{m}$, $a(\tilde{m})$ minimizes $\mathcal{R}(a(\tilde{m}),\ell;\tilde{m})$. Note that 
\begin{equation}
\mathcal{R}_{aa}(a,m,\tilde{m})=\int_{-\infty}^{\infty}{I(q)L''(a-q)g(q|\tilde{m})dq}=\int_{-\infty}^{\infty}{I(q)g(q|\tilde{m})dq}>0.
\end{equation}
Let $\tilde{m}\in\mathbb{R}$. The solution of the first-order condition is the unique minimizer:
\begin{equation}
0=\mathcal{R}_{a}(\tilde{m},a,m)=\int_{-\infty}^{\infty}{I(q)L'(a-q)g(q|\tilde{m})dq}\:\Rightarrow\:a(\tilde{m})=\frac{\int_{-\infty}^{\infty}{qI(q)g(q|\tilde{m})dq}}{\int_{-\infty}^{\infty}{I(q)g(q|\tilde{m})dq}}
\end{equation}
and hence
\begin{equation}
a'(\tilde{m})=\frac{\int_{-\infty}^{\infty}{\int_{-\infty}^{\infty}{sI(s)I(t)(m(s)-m(t))g(s|\tilde{m})g(t|\tilde{m})dsdt}}}{\left(\int_{-\infty}^{\infty}{I(r)g(r|\tilde{m})dr}\right)^{2}}.
\end{equation}
\end{proof}

%%%%%%%%%%%%%%%%%%%%%%%%%%%%%%%%%%%%%%%%%
%%% PERFECT BAYESIAN-NASH EQUILIBRIUM %%%
%%%%%%%%%%%%%%%%%%%%%%%%%%%%%%%%%%%%%%%%%
\begin{proposition}
The equilibrum is
\end{proposition}
\begin{proof}
\begin{align}
0&=\int_{-\infty}^{\infty}{\left(\frac{\int_{-\infty}^{\infty}{sI(s)g(s|m(q)+e)ds}}{\int_{-\infty}^{\infty}{I(t)g(t|m(q)+e)dt}}-q\right)\left(\frac{\int_{-\infty}^{\infty}{\int_{-\infty}^{\infty}{sI(s)I(t)(m(s)-m(t))g(s|m(q)+e)g(t|m(q)+e)dsdt}}}{\left(\int_{-\infty}^{\infty}{I(r)g(r|m(q)+e)dr}\right)^{2}}\right)f(e)de}\\
&=
\end{align}
\end{proof}
\end{landscape}
\end{document}