\documentclass[12pt]{article}
\usepackage{amsmath,amssymb,amsfonts,amsthm,fullpage,setspace,float,graphicx}
\usepackage[round]{natbib}
\newtheorem{corollary}{Corollary}
\newtheorem{lemma}{Lemma}
\newtheorem{definition}{Definition}
\newtheorem{proposition}{Proposition}
\renewcommand{\qedsymbol}{$\blacksquare$}
\DeclareMathOperator*{\Beta}{Beta}
\DeclareMathOperator*{\argmin}{argmin}
\DeclareMathOperator*{\argmax}{argmax}
\begin{document}
\noindent\textit{I've done the math enough to know the dangers of our second guessing\\ Doomed to crumble unless we grow, and strengthen our communication}\\ \\$\sim$ From ``Schism,'' by Tool. \\ \\
%%%%%%%%%%%%%%%%%%%%%%%%%%%%%%%%%%%%%%%%
%%% UNIFORM NOISE AND QUADRATIC LOSS %%%
%%%%%%%%%%%%%%%%%%%%%%%%%%%%%%%%%%%%%%%%
\noindent\underline{Uniform Noise and Quadratic Loss}. Set $L(x)=\frac{1}{2}x^2$ and $g(q)=\mathbf{1}_{0\leq q\leq1}$. Fix $\bar{\epsilon}\in(0,1)$, and $I\in L^{2}[0,1]$. Action rules are maps from \textit{received messages} to actions: $\tilde{m}\mapsto a(\tilde{m})$. Message rules are maps from states to \textit{sent messages}: $q\mapsto m(q)$. Sent and received messages are related by the identity $\tilde{m}=m(q)+e$, where $e\sim U[-\bar{\epsilon},\bar{\epsilon}]$. Let $f$ denote the density of $e$. $q$ is known to the sender, but not the receiver; $e$ is known to neither the sender, nor the receiver. We take as the ambient action function space $\mathcal{A}:=L^{2}[\underline{m}-\bar{\epsilon},\overline{m}+\bar{\epsilon}]$ and the ambient message function space $\mathcal{M}:=L^{2}[\underline{m},\overline{m}]$.
% action rule
%%%%%%%%%%%%%%%%%%%%%%%%%%%%%%%
%%% DEFINITION: ACTION RULE %%%
%%%%%%%%%%%%%%%%%%%%%%%%%%%%%%%
\begin{definition}
A map $a\in \mathcal{A}$ is an action function if $a([\underline{m}-\bar{\epsilon},\overline{m}+\bar{\epsilon}])\subseteq[\underline{m},\overline{m}]$, and if $a$ is weakly increasing.  
\end{definition}
%%%%%%%%%%%%%%%%%%%%%%%%%%%%%%%%
%%% DEFINITION: MESSAGE RULE %%%
%%%%%%%%%%%%%%%%%%%%%%%%%%%%%%%%
\noindent Message functions are similarly defined, but with the additional requirement that they satisfy certain boundary conditions.
\begin{definition}
A map $m\in \mathcal{M}$ is a message function if $m([\underline{m},\overline{m}])\subseteq[\underline{m},\overline{m}]$, $m(0)=\underline{m}$, $m(1)=\overline{m}$, and $m$ is weakly increasing. 
\end{definition}
\noindent The boundary conditions $m(0)=\underline{m}$ and $m(1)=\overline{m}$ are made to quarantee the uniquesness of an equilibrium. Note that the receiver's action has domain $[\underline{m}-\bar{\epsilon},\overline{m}+\bar{\epsilon}]$. The sender's message has codomain $[\underline{m},\overline{m}]$, but it is corrupted by noise. Although a \textit{sent} message may be no smaller than $\underline{m}$, and no larger than $\overline{m}$, a \textit{received} message may be as small as $\underline{m}-\bar{\epsilon}$, or as large as $\overline{m}+\bar{\epsilon}$. The receiver must choose an action in $[\underline{m},\overline{m}]$ for messages less than $\underline{m}$ or greater than $\overline{m}$.\\

%%%%%%%%%%%%%%%%%%%%%%%%%%%%%%%%%%%%%%%%%%%%%%%%%%%%%%%%%
%%% STRICTLY INCREASING, CONTINUOUS ACTION FUNCTIONS %%%%
%%%%%%%%%%%%%%%%%%%%%%%%%%%%%%%%%%%%%%%%%%%%%%%%%%%%%%%%%
\noindent\underline{Continuous, Strictly Increasing Action and Message Functions}. In what follows, I show that there does not exist an equilibrium in which the action and message functions are continuous and strictly increasing. By way of contradiction, suppose for the remainder of this section that action and message functions are continuous and strictly increasing. Since $m$ is continuous and strictly increasing, so too is $\ell$. \\
\begin{proposition}
There does not exist an equilibrium in which the message function is strictly increasing and continuous.
\end{proposition}
\begin{proof}
Suppose not. Let $m$ denote the equilibrium message function and $\ell$ its inverse. 

%%%%%%%%%%%%%%%
%%% BELIEFS %%%
%%%%%%%%%%%%%%%
\noindent\textit{Beliefs.} At the start of the game, the receiver's prior over states is $g$. Let $g(q|\tilde{m})$ denote her posterior having received the message $\tilde{m}\in[\underline{m}-\bar{\epsilon},\overline{m}+\bar{\epsilon}]$. Define 
\begin{align}
\overline{\ell}(\tilde{m})&=\ell(\min\{\tilde{m}+\bar{\epsilon},\overline{m}\}),\text{ and}\\
\underline{\ell}(\tilde{m})&=\ell(\max\{\tilde{m}-\bar{\epsilon},\underline{m}\}).
\end{align} 
Given a received message $\tilde{m}$, $\overline{\ell}(\tilde{m})$ is the \textit{largest} possible state that could have resulted in $\tilde{m}$, while $\underline{\ell}(\tilde{m})$ is the \textit{smallest} possible state that could have resulted in $\tilde{m}$. Observe that for $\tilde{m}\in(\underline{m}-\bar{\epsilon},\overline{m}+\bar{\epsilon})$, 
\begin{equation}
\min\{\tilde{m}+\bar{\epsilon},\overline{m}\}>\max\{\tilde{m}-\bar{\epsilon},\underline{m}\}\Rightarrow\ell(\min\{\tilde{m}+\bar{\epsilon},\overline{m}\})>\ell(\max\{\tilde{m}-\bar{\epsilon},\underline{m}\})\Rightarrow\overline{\ell}(\tilde{m})>\underline{\ell}(\tilde{m}).
\end{equation}
$\overline{\ell}(\underline{m}-\bar{\epsilon})=\underline{\ell}(\underline{m}-\bar{\epsilon})$, and $\overline{\ell}(\underline{m}+\bar{\epsilon})=\underline{\ell}(\underline{m}+\bar{\epsilon})$

By Baye's Law,
\begin{equation}
g(q|\tilde{m})=\frac{f(\tilde{m}|q)g(q)}{f(\tilde{m})}=\frac{\mathbf{1}_{-\bar{\epsilon}\leq\tilde{m}-m(q)\leq\bar{\epsilon}}\cdot\mathbf{1}_{0\leq q\leq 1}}{\int_{-\infty}^{\infty}{\mathbf{1}_{-\bar{\epsilon}\leq\tilde{m}-m(t)\leq\bar{\epsilon}}\cdot\mathbf{1}_{0\leq t\leq 1}}dt}=\frac{\mathbf{1}_{\underline{\ell}(\tilde{m})\leq q\leq\overline{\ell}(\tilde{m})}}{\overline{\ell}(\tilde{m})-\underline{\ell}(\tilde{m})}.
\end{equation}

%%%%%%%%%%%%%%%
%%% PAYOFFS %%%
%%%%%%%%%%%%%%%
\noindent\textit{Payoffs.} The sender's and receiver's payoffs are 
\begin{align}
S(a,m;q)&=\int_{m(q)-\bar{\epsilon}}^{m(q)+\bar{\epsilon}}{I(q)L(a(\tilde{m})-q)f(\tilde{m})d\tilde{m}}\\
R(a,m;\tilde{m})&=\int_{0}^{1}{I(q)L(a(\tilde{m})-q)g(q|\tilde{m})dq}
\end{align}
respectively.\\

%%%%%%%%%%%%%%%%%%%%%%%%%%%%
%%% THE SENDER'S PROBLEM %%% 
%%%%%%%%%%%%%%%%%%%%%%%%%%%%
\begin{lemma}\label{lemma:message}
The sender's best-response, $\mathcal{S}:\mathcal{A}\rightarrow\mathcal{M}$, is given by
\begin{equation}
[\mathcal{S}a](m)=\frac{a(m+\bar{\epsilon})+a(m-\bar{\epsilon})}{2}.
\end{equation}
\end{lemma}
\begin{proof}
Fix $q\in[0,1]$ and put $r_{1}(m)=a(m+\bar{\epsilon})-a(m-\bar{\epsilon})$, and $r_{2}(m)=(m+\bar{\epsilon})+a(m-\bar{\epsilon})-2q$. Observe that
\begin{align}
S_{m}(a,m;q)&=L(a(m+\bar{\epsilon})-q)-L(a(m-\bar{\epsilon})-q)\\
&=\frac{1}{2}(a(m+\bar{\epsilon})-q)^{2}-\frac{1}{2}(a(m-\bar{\epsilon})-q)^{2}\\
&=\frac{1}{2}(a(m+\bar{\epsilon})-a(m-\bar{\epsilon}))(a(m+\bar{\epsilon})+a(m-\bar{\epsilon})-2q)\\
&=(a(m+\bar{\epsilon})-a(m-\bar{\epsilon}))(E[a(m+\epsilon)]-q)\\
&=r_{1}(m)r_{2}(m).
\end{align}
Since $a$ is weakly increasing, $r_{1}(m)$ is non-negative, and $r_{2}$ is weakly increasing. Put
\begin{equation}
E[a(m+e)]
\end{equation}
There are three cases. 
\begin{enumerate}
\item If $0\leq q\leq E[a(\underline{m}+\epsilon)]$, then $r_{2}(\underline{m})\geq0$, and hence $m(q)=\underline{m}$. 
\item $E[a(\underline{m}+\epsilon)]<q<E[a(\overline{m}+\epsilon)]$, and hence $r_{2}(\underline{m})<0<r_{2}(\overline{m})$, By the intermediate value theorem, there is an $m$ such that $r_{2}(m)=0$. Since $r_{2}$ is increasing, $m$ is a minimizer. 
\item If $0<r_{2}(\underline{m})<r_{2}(\overline{m})$, then $r_{2}(\overline{m})\leq0$, and hence $m(q)=\overline{m}$
\end{enumerate}
\end{proof}

%%%%%%%%%%%%%%%%%%%%%%%%%%%%%%
%%% THE RECEIVER'S PROBLEM %%%
%%%%%%%%%%%%%%%%%%%%%%%%%%%%%%
\begin{lemma}
The receiver's best-response, $\mathcal{R}:\mathcal{M}\rightarrow\mathcal{A}$, is given by
\begin{equation}
[\mathcal{R}\ell](\tilde{m})=\left[\int_{\underline{\ell}(\tilde{m})}^{\overline{\ell}(\tilde{m})}{I(r)dr}\right]^{-1}\left[\int_{\underline{\ell}(\tilde{m})}^{\overline{\ell}(\tilde{m})}{rI(r)dr}\right].
\end{equation}
if $\overline{\ell}(\tilde{m})>\underline{\ell}(\tilde{m})$, and $[\mathcal{R}\ell](\tilde{m})=\overline{\ell}(\tilde{m})$ otherwise. 
\end{lemma}
\begin{proof}
Let $\tilde{m}\in\mathbb{R}$. $\overline{\ell}(\tilde{m})\geq\underline{\ell}(\tilde{m})$. If $\overline{\ell}(\tilde{m})>\underline{\ell}(\tilde{m})$, then $R$ is strictly convex in its first argument:
\begin{equation}
R_{aa}(a,m,\tilde{m})=\frac{1}{\overline{\ell}(\tilde{m})-\underline{\ell}(\tilde{m})}\int_{\underline{\ell}(\tilde{m})}^{\overline{\ell}(\tilde{m})}{I(q)L''(a-q)dq}>0.
\end{equation}
The solution of the first-order condition is the unique minimizer:
\begin{equation}
0=R_{a}(\tilde{m},a,m)=\frac{1}{\overline{\ell}(\tilde{m})-\underline{\ell}(\tilde{m})}\int_{\underline{\ell}(\tilde{m})}^{\overline{\ell}(\tilde{m})}{I(q)L'(a-q)dq}
\end{equation}
which implies that
\begin{equation}
a(\tilde{m})=\left[\int_{\ell(\tilde{m}-\bar{\epsilon})}^{\ell(\tilde{m}+\bar{\epsilon})}{I(r)dr}\right]^{-1}\left[\int_{\ell(\tilde{m}-\bar{\epsilon})}^{\ell(\tilde{m}+\bar{\epsilon})}{rI(r)dr}\right].
\end{equation}
If $\overline{\ell}(\tilde{m})=\underline{\ell}(\tilde{m})$, then the receiver knows with probability $1$ that $q=\overline{\ell}(\tilde{m})$, and she finds $a(\tilde{m})=\overline{\ell}(\tilde{m})=\underline{\ell}(\tilde{m})$ optimal.
\end{proof}
\end{proof}

%%%%%%%%%%%%%%%%%%%%%%%%%%%%%%%%%%%%%%%%%%%%%%%%%%%%%%%%%%%%%%%
%%% THERE IS A UNIQUE SOLUTION TO THE OPTIMALITY CONDITIONS %%%
%%%%%%%%%%%%%%%%%%%%%%%%%%%%%%%%%%%%%%%%%%%%%%%%%%%%%%%%%%%%%%%
\begin{proposition}
There exists a unique $\ell^{*}\in X$ such that $\mathcal{R}(\mathcal{S}(\ell^{*}))=\ell^{*}$. 
\end{proposition}
\begin{proof}
The following calculation shows that $||\overline{u}-\overline{v}||_{Y}=||u-v||_{X}$:
\begin{align}
||\overline{u}-\overline{v}||_{Y}^{2}&=\int_{\underline{m}-\bar{\epsilon}}^{\overline{m}+\bar{\epsilon}}{|u(\min\{\tilde{m}+\bar{\epsilon},\overline{m}\})-v(\min\{\tilde{m}+\bar{\epsilon},\overline{m}\})|^{2}d\tilde{m}}\\
&=\int_{\underline{m}}^{\overline{m}+2\bar{\epsilon}}{|u(\min\{\tilde{m},\overline{m}\})-v(\min\{\tilde{m},\overline{m}\})|^{2}d\tilde{m}}\\
&=\int_{\underline{m}}^{\overline{m}}{|u(\tilde{m})-v(\tilde{m})|^{2}d\tilde{m}}+\int_{\overline{m}}^{\overline{m}+2\bar{\epsilon}}{|u(\overline{m})-v(\overline{m})|^{2}d\tilde{m}}\\
&=||u-v||_{X}^{2}.
\end{align}
A similar calculation shows that $||\underline{u}-\underline{v}||_{Y}=||u-v||_{X}$. First consider $\mathcal{R}$. Choose $f,g\in X$. 
\begin{align}
||\mathcal{S}(f)-\mathcal{S}(g)||_{Y}^{2}&=\int_{\underline{m}}^{\overline{m}}{\left|\left(\frac{f(m+\bar{\epsilon})}{2}+\frac{f(m-\bar{\epsilon})}{2}\right)-\left(\frac{g(m+\bar{\epsilon})}{2}+\frac{g(m-\bar{\epsilon})}{2}\right)\right|^{2}dm}\\
&=\int_{\underline{m}}^{\overline{m}}{\left|\left(\frac{f(m+\bar{\epsilon})}{2}-\frac{g(m+\bar{\epsilon})}{2}\right)+\left(\frac{f(m-\bar{\epsilon})}{2}-\frac{g(m-\bar{\epsilon})}{2}\right)\right|^{2}dm}\\
&\leq\frac{1}{2}\int_{\underline{m}}^{\overline{m}}{|f(m+\bar{\epsilon})-g(m+\bar{\epsilon})|^{2}dm}+\frac{1}{2}\int_{\underline{m}}^{\overline{m}}{|f(m-\bar{\epsilon})-g(m-\bar{\epsilon})|^{2}dm}\\
&\leq\frac{1}{2}\int_{\underline{m}}^{\overline{m}}{|f(m)-g(m)|^{2}dm}+\frac{1}{2}\int_{\underline{m}}^{\overline{m}}{|f(m)-g(m)|^{2}dm}\\
&=||f-g||_{X}
\end{align}
Now consider $\mathcal{S}$. Choose $u,v\in Y$. 
\begin{align}
||\mathcal{S}(u)-\mathcal{S}(v)||_{Y}^{2}=\int_{\underline{m}}^{\overline{m}}{}.
\end{align}
Now $\mathcal{R}\circ\mathcal{S}:X\rightarrow X$ contracts:
\begin{equation}
||\mathcal{R}(\mathcal{S}(f))-\mathcal{R}(\mathcal{S}(g))||_{X}\leq||\mathcal{S}(f)-\mathcal{S}(g)||_{Y}\leq c||f-g||_{X}. 
\end{equation}
The result obtains from the contraction mapping theorem. 
\end{proof}

%%%%%%%%%%%%%%%%%%%%%%%%%%%%%%%%%%%%%%%%%%%%%%%%%%%%%%%%%%%%%%%%%%%%
%%% THE UNIQUE SOLUTION TO THE OPTIMALITY CONDITIONS IS DISCRETE %%%
%%%%%%%%%%%%%%%%%%%%%%%%%%%%%%%%%%%%%%%%%%%%%%%%%%%%%%%%%%%%%%%%%%%%
\begin{proposition}
$\ell^{*}$ is piecewise constant.
\end{proposition}
\begin{proof}
Consider $x_{k-1}<x_{k}<x_{k+1}$. Let $(x_{k-1},x_{k},x_{k+1})\mapsto F(x_{k-1},x_{k},x_{k+1})$, where
\begin{equation}
F(x_{k-1},x_{k},x_{k+1})=-x_{k}+\frac{1}{2}\left(\frac{\int_{x_{k}}^{x_{k+1}}{rI(r)dr}}{\int_{x_{k}}^{x_{k+1}}{I(r)dr}}+\frac{\int_{x_{k-1}}^{x_{k}}{rI(r)dr}}{\int_{x_{k-1}}^{x_{k}}{I(r)dr}}\right).
\end{equation}
$F$ is strictly increasing in its first and third arguments:
\begin{align}
F_{1}(x_{k-1},x_{k},x_{k+1})&=\frac{I(x_{k-1})\int_{x_{k-1}}^{x_{k}}{(r-x_{k-1})I(r)dr}}{\left(\int_{x_{k-1}}^{x_{k}}{rI(r)dr}\right)^{2}}>0;\\
F_{3}(x_{k-1},x_{k},x_{k+1})&=\frac{I(x_{k+1})\int_{x_{k}}^{x_{k+1}}{(x_{k+1}-r)I(r)dr}}{\left(\int_{x_{k}}^{x_{k+1}}{rI(r)dr}\right)^{2}}>0.
\end{align}
Two applications of l'H\^{o}pital's rule yields
\begin{align}
\lim_{x_{k-1}\uparrow x_{k}}F(x_{k-1},x_{k},x_{k+1})&\overset{H}{=}-x_{k}+\frac{1}{2}\left(\frac{\int_{x_{k}}^{x_{k+1}}{rI(r)dr}}{\int_{x_{k}}^{x_{k+1}}{I(r)dr}}+x_{k}\right)\\
\lim_{x_{k+1}\downarrow x_{k}}F(x_{k-1},x_{k},x_{k+1})&\overset{H}{=}-x_{k}+\frac{1}{2}\left(x_{k}+\frac{\int_{x_{k-1}}^{x_{k}}{rI(r)dr}}{\int_{x_{k-1}}^{x_{k}}{I(r)dr}}\right).
\end{align}
Therefore, there is a unique $x_{k+1}(x_{k-1},x_{k})\in(x_{k},\infty)$ such that $F(x_{k-1},x_{k},x_{k+1}(x_{k-1},x_{k}))=0$.
\end{proof}

%%%%%%%%%%%%%%%%%%%%%%%%%%%%%%%%%%%%%%%%%%%
%%% PIECEWISE CONSTANT ACTION FUNCTIONS %%%
%%%%%%%%%%%%%%%%%%%%%%%%%%%%%%%%%%%%%%%%%%%
\noindent\underline{Piecewise Constant Action Functions.}

%%%%%%%%%%%%%%%%%%%%%%%%%%%%%%%%%%%%%%%%
%%% DEFINITION: DISCRETE EQUILIBRIUM %%%
%%%%%%%%%%%%%%%%%%%%%%%%%%%%%%%%%%%%%%%%
\begin{definition}
An equilibrium $(a,m)$ is said to be discrete if there is $n\in\mathbb{N}$, a partition $P=\{x_{0},x_{1},\ldots,x_{n}\}$ of $[0,1]$ with $0=x_{0}<x_{1}<\cdots<x_{n}=1$, a set of actions $A=\{a_{0},a_{1},\ldots,a_{n}\}\subset[0,1]$, and a set of messages $M=\{m_{0},m_{1},\ldots,m_{N}\}$, such that for each $q\in(0,1)$,
\begin{equation*}
m(q)=\sum_{t=1}^{n}m_{t}\mathbf{1}_{x_{t-1}\leq q<x_{t}}
\end{equation*}
and $a$ is a bijection from $M$ to $A$.
\end{definition}

%%%%%%%%%%%%%%%
%%% BELIEFS %%%
%%%%%%%%%%%%%%%
\noindent\textit{Beliefs.} At the start of the game, the receiver's prior over states is $g$. Let $g(q|\tilde{m})$ denote her posterior having received the message $\tilde{m}\in[\underline{m}-\bar{\epsilon},\overline{m}+\bar{\epsilon}]$. With discrete messages, there is no reason for the sender to be misinterpreted. The receiver perfectly infers the message sent.\\

%%%%%%%%%%%%%%%
%%% PAYOFFS %%%
%%%%%%%%%%%%%%%
\noindent\textit{Payoffs.} The sender's and receiver's payoffs are
\begin{align}
S(a,m;q)&=I(q)L(a(m(q))-q)\\
R(a,m;\tilde{m})&=\int_{0}^{1}{I(q)L(a(\tilde{m})-q)g(q|\tilde{m})dq}.
\end{align}

%%%%%%%%%%%%%%%%%%%%%%%%%%%%
%%% THE SENDER'S PROBLEM %%% 
%%%%%%%%%%%%%%%%%%%%%%%%%%%%
\begin{lemma}
The sender's best-response, $\mathcal{S}:\mathcal{A}\rightarrow\mathcal{M}$, is given by
\end{lemma}
\begin{proof}

\end{proof}

%%%%%%%%%%%%%%%%%%%%%%%%%%%%%%
%%% THE RECEIVER'S PROBLEM %%%
%%%%%%%%%%%%%%%%%%%%%%%%%%%%%%
\begin{lemma}
The receiver's best-response, $\mathcal{R}:\mathcal{M}\rightarrow\mathcal{A}$, is given by
\begin{equation}
a_{t}=\left[\int_{x_{t-1}}^{x_{t}}{I(q)dq}\right]^{-1}\left[\int_{x_{t-1}}^{x_{t}}{qI(q)dq}\right]
\end{equation}
\end{lemma}
\begin{proof}

\end{proof}

%%%%%%%%%%%%%%%%%%%%%%%%%%%%%%%%%%%%%%%%%%%%%%%%%%%%%%%%%%%%%%%
%%% PROPOSITION: THERE EXISTS A UNIQUE DISCRETE EQUILIBRIUM %%%
%%%%%%%%%%%%%%%%%%%%%%%%%%%%%%%%%%%%%%%%%%%%%%%%%%%%%%%%%%%%%%%
\begin{proposition}
There exists a unique discrete equilibrium.
\end{proposition}
\begin{proof}
x
\end{proof}
\end{document}