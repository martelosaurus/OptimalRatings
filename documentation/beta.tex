\documentclass{article}
\usepackage{amsmath,amsfonts}
\begin{document}

\noindent\textit{I've done the math enough to know the dangers of our second guessing}\\
\textit{Doomed to crumble unless we grow and strengthen our communication}\\

\hspace{.8\textwidth}$\sim$ Tool

% -----------------------------------------------------------------------------
% THE MODEL
\section{The Model}

There are two players, a sender and a receiver, and a state of nature $q\in [ 0,1]$ that is known to the sender but not the receiver. Let $g$ denote the density of $q$. The sender sends an \textit{intended} message $m(q)\in [ 0,1]$. The receiver receives a noisy version of the intended message, which we call the \textit{received} message, $\widetilde{m} =m(q)+e$. $e$ is distributed according to a continuous density $ f$ with support $[-\overline{e },\overline{e }]$, where for some integer $N\geq2$, $\overline{e}=\frac{1}{2N}$. 

We specify $f$ as follows. Let $x:[-\overline{e},\overline{e}]\rightarrow[0,1]$ be given by
\begin{equation}
	x(e)=\frac{1}{2}\left[1+\frac{e}{\overline{e}}\right]
\end{equation}
Consider, as an example, the Beta PDF:
\begin{equation}
	f(e)=x(e)^{a-1}(1-x(e))^{b-1}
\end{equation}
for constants $a>0$ and $b>0$. The receiver then takes an action $a(\widetilde{m})\in[0,1]$. Both sender and receiver have utility over the state, $q$, and the receiver's action, $a$, of 
\begin{equation}
	U(q,a)=-\frac{1}{2}(q-a)^{2}
\end{equation}
Prior to the start of the game, the sender can specify an intended message function $m(q)$ that she will use. The receiver chooses an action based upon the function $m(q)$ and the received message $\widetilde{m}$, and we denote this function $a(\widetilde{m})$. We work backward and start with the optimal action, given $m(q)$ and $\widetilde{m}$.

Fix a message function $m$. Let $Q(\widetilde{m})$ denote the set of all states $q\in [ 0,1]$ such that for some noise $e\in [ -\bar{e},\bar{e}]$, $\widetilde{m}=m(q)+e$. In set builder notation,
\begin{equation}
	Q(\widetilde{m})=\{q\in[0,1]\:|\:m(q)-\bar{e}\leq \widetilde{m}\leq m(q)+\bar{e}\}.
\end{equation}
Finally, define
\begin{align}
	q_{+}(\widetilde{m})&\equiv \sup Q(\widetilde{m})\\
	q_{-}(\widetilde{m})&\equiv \inf Q(\widetilde{m})
\end{align}
$q_{+}(\widetilde{m})$ and $ q_{-}(\widetilde{m})$ are the highest and lowest states that could possibly be associated with the received message $\widetilde{m}$. Define
\begin{equation}
	I(\widetilde{m};\alpha,\beta,\gamma)\equiv\int_{\alpha}^{\beta}{q^{\gamma}f(\tilde{m}-m(q))dq}
\end{equation}
Suppose that the receiver receives the message $\widetilde{m}\in [ - \bar{e},1+\bar{e}]$. $q|\widetilde{m}$ has support $[q_{-}( \widetilde{m}),q_{+}(\widetilde{m})]$ and density
\begin{equation}\label{eq:posterior} 
	g(q|\widetilde{m})=\frac{f(\widetilde{m}-m(q))g(q)}{\int_{0}^{1}{ f(\widetilde{m}-m(t))g(t)dt}}  
\end{equation} 
The receiver's problem is to 
\begin{equation} 
	\max_{a(\widetilde{m})}\int_{q_{-}(\widetilde{m})}^{q_{+}(\widetilde{m})}{U(q,a)g(q|\widetilde{m})}dq.  
\end{equation} 
The receiver's optimal action is simply the expected value of the state, $q$, given the received message $\widetilde{m}$ : 
\begin{equation}
	a(\widetilde{m})=\int_{q_{-}(\widetilde{m})}^{q_{+}(\widetilde{m})}{qg(q| \widetilde{m})dq}=\frac{I(\widetilde{m};q_{-}(\widetilde{m}),q_{+}(\widetilde{m}),1)}{I(\widetilde{m};q_{-}(\widetilde{m}),q_{+}(\widetilde{m}),0)}\label{eq:optact} 
\end{equation}


It will be helpful to refer to the \textit{cost }of a message function. Let the cost functional $C$ be given by 
\begin{equation}
	C[m]\equiv \int_{0}^{1}{\int_{-\bar{e}}^{\bar{e}}{(q-a(m(q)+e))^{2}f(e)de}dq},  
\end{equation}
where $a$ is the receiver's optimal action from Equation (\ref{eq:optact}). $C[m]$ is the expected loss for a given message function $m$. The integrand is the loss for a given state $q$ and action $a(\widetilde{m})$. The interior integral integrates over the possible exogenous errors, to generate the expected loss given the state. The exterior integral integrates over possible states. Therefore, the \textit{sender's problem} is to choose a message function $m$ that minimizes $C[m]$: 
\begin{equation}
	\min_{m\in M}C[m]
\end{equation}
where $M$ is the space of weakly increasing piece-wise continuous functions on $[0,1]$. The change of variables $\widetilde{m}=m(q)+e$ and an application of Fubini's Theorem yield 
\begin{align}
	C[m]& =\int_{0}^{1}{\int_{m(q)-\bar{e}}^{m(q)+\bar{e}}{(q-a( \widetilde{m}))^{2}f(\widetilde{m}-m(q))g(q)d\widetilde{m}}dq} \\
	& =\int_{-\bar{e}}^{1+\bar{e}}{\int_{q_{-}(\widetilde{m})}^{q_{+}(\widetilde{m})}{(q-a(\widetilde{m}))^{2}f(\widetilde{m}-m(q))g(q)dq}d\widetilde{m}}.
\end{align}
We consider the costs of identity (``$\mathcal{I}$'') and discrete (``$\mathcal{D}$'') message functions.

% -----------------------------------------------------------------------------
% IDENTITY MESSAGE
\section{Identity Message}
We consider the identity message function, $m(q)=q$, only for the case in which $q$ is uniformly distributed. 
\begin{equation}
	a(\widetilde{m})=
	\begin{cases}
		\overline{a}(\widetilde{m}) & \text{ if } 1-\overline{e}<\widetilde{m}\leq 1+\overline{e}\\
		a(\widetilde{m}) & \text{ if } \overline{e}<\widetilde{m}\leq 1-\overline{e}\\
		\underline{a}(\widetilde{m}) & \text{ if } -\overline{e}\leq\widetilde{m}\leq \overline{e}
	\end{cases}
\end{equation}
where
\begin{align}
	\overline{a}(\widetilde{m})&=\frac{I(\widetilde{m};\widetilde{m}-\bar{e},1,1)}{I(\widetilde{m};\widetilde{m}-\bar{e},1,0)}\\
	a(\widetilde{m})&=\frac{I(\widetilde{m};\widetilde{m}-\bar{e},\widetilde{m}+\bar{e},1)}{I(\widetilde{m};\widetilde{m}-\bar{e},\widetilde{m}+\bar{e},0)}\\
	\underline{a}(\widetilde{m})&=\frac{I(\widetilde{m};0,\widetilde{m}+\bar{e},1)}{I(\widetilde{m};0,\widetilde{m}+\bar{e},0)}
\end{align}
Note that the normalizing constant cancels out when computing the conditional expectation. The cost of the identity message function is given by
\begin{align}
	C[m_{\mathcal{I}}]&=\int_{-\bar{e}}^{1+\bar{e}}{\int_{q_{-}(\widetilde{m})}^{q_{+}(\widetilde{m})}{(q-a(\widetilde{m}))^{2}f(\widetilde{m}-q)dq}d\widetilde{m}}.
\end{align}
where $q_{+}(\widetilde{m})=\min\{\widetilde{m}+\bar{e},1\}$ and $q_{-}(\widetilde{m})=\max\{\widetilde{m}-\bar{e},0\}$. Define
\begin{align}
	\overline{z}&=\int_{1-\bar{e}}^{1+\bar{e}}{\int_{\widetilde{m}-\bar{e}}^{1}{(q-\overline{a}(\widetilde{m}))^{2}f(\widetilde{m}-q)dq}d\widetilde{m}}\\
	z&=\int_{\bar{e}}^{1-\bar{e}}{\int_{\widetilde{m}-\bar{e}}^{\widetilde{m}+\bar{e}}{(q-a(\widetilde{m}))^{2}f(\widetilde{m}-q)dq}d\widetilde{m}}\\
	\underline{z}&=\int_{-\bar{e}}^{\bar{e}}{\int_{0}^{\widetilde{m}+\bar{e}}{(q-\underline{a}(\widetilde{m}))^{2}f(\widetilde{m}-q)dq}d\widetilde{m}}
\end{align}
so that $C[m_{\mathcal{I}}]=\overline{z}+z+\underline{z}$.
% -----------------------------------------------------------------------------
% DISCRETE MESSAGE 
\section{Discrete Message}
Fix an integer $M\geq1$ and define $K=M\times N$. Consider the partition
\begin{equation}\label{eq:partition}
	0=x_{0}<x_{1}<\cdots<x_{K}<x_{K+1}=1
\end{equation}
of $[0,1]$. For each $i\in\{0,\ldots,K\}$, define $X_{i}=[x_{i},x_{i+1})$. A discrete message with $K+1$ messages is given by
\begin{equation}
	m_{\mathcal{D}}(q)=\frac{1}{K}\sum_{i=0}^{K}{\chi_{X_{i}}(q)}.
\end{equation}
Let $j_{+},j_{-}:[-\bar{e},1+\bar{e}]\rightarrow\{0,\ldots,K\}$ be given by
\begin{align}
	j_{+}(\widetilde{m})&=\min\{\lfloor \widetilde{m}K\rfloor+M,K\}\\
	j_{-}(\widetilde{m})&=\max\{0,\lfloor \widetilde{m}K\rfloor-M\}
\end{align}
Note that 
\begin{align}
	q_{+}(\widetilde{m})&=x_{j_{+}(\widetilde{m})}\\
	q_{-}(\widetilde{m})&=x_{j_{-}(\widetilde{m})}
\end{align}
The action 
\begin{equation}
	a(\widetilde{m})=\frac{I(\widetilde{m};x_{j},x_{j+1},1)}{I(\widetilde{m};x_{j},x_{j+1},0)}
\end{equation}
The cost is given by 
\begin{align}
	C[m_{\mathcal{D}}]&=\int_{-\bar{e}}^{1+\bar{e}}{\int_{x_{j_{-}(\widetilde{m})}}^{x_{j_{+}(\widetilde{m})}}{(q-a(\widetilde{m}))^{2}f(\widetilde{m}-m_{\mathcal{D}}(q))g(q)dq}d\widetilde{m}}.\\
	&=\sum_{j=0}^{K}{\int_{x_{j}}^{x_{j}}{(q-a(\widetilde{m}))^{2}f(\widetilde{m}-j/K)g(q)dq}}.\\
\end{align}
It is sufficient to choose the partition $x_{0},\ldots,x_{K+1}$ in Equation \ref{eq:partition}. The first-order for $x_{j}$ is 
\begin{equation}
	0=\sum_{j=0}^{M}{\frac{I(\widetilde{m};x_{j},x_{j+1},1)}{I(\widetilde{m};x_{j},x_{j+1},0)}}
\end{equation}
\end{document}
